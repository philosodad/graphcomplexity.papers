\documentclass[conference, 11pt]{IEEEtran} 
\usepackage{verbatim}
\usepackage{multirow} \usepackage{enumerate}
\usepackage{amsmath,enumerate} \usepackage{amsthm}
\usepackage{algorithm}
\usepackage{algorithmic}
\usepackage{pstricks}
\usepackage{amssymb, latexsym}
\usepackage{graphicx}
\DeclareGraphicsRule{*}{mps}{*}{}

\usepackage{pgf}
\usepackage{tikz}
\usetikzlibrary{decorations.pathmorphing} % LATEX and plain TEX when using Tik Z
\usetikzlibrary{positioning}
\usetikzlibrary{er}
\tikzstyle{vx}=[draw,circle,fill=black!50,minimum size=2pt, inner sep=1pt, node distance=15mm]
\tikzstyle{bup}=[semithick, decoration={bent, aspect=.3, amplitude=4}, decorate, ->, >=stealth]
\tikzstyle{bdn}=[semithick, decoration={bent, aspect=.3, amplitude=-4}, decorate, ->, >=stealth]
\tikzstyle{BUP}=[thick, decoration={bent, aspect=.3, amplitude=8}, decorate, ->, >=stealth]
\tikzstyle{BDN}=[thick, decoration={bent, aspect=.3, amplitude=-8}, decorate, ->, >=stealth]
\tikzstyle{MUP}=[thick, decoration={bent, aspect=.3, amplitude=16}, decorate, ->, >=stealth]
\tikzstyle{MDN}=[thick, decoration={bent, aspect=.3, amplitude=-16}, decorate, ->, >=stealth]
\tikzstyle{str}=[semithick, decorate, ->, >=stealth]
\tikzstyle{cr}=[draw, circle, fill=black!25,minimum size=150pt]

% \paperheight=11in \paperwidth=8.5in \textheight=9.0in
% \textwidth=6.5in \voffset=-.875in \hoffset=-.875in
\newenvironment{code} {\begin {quote}\begin{footnotesize}}
    {\end{footnotesize}\end{quote}}

% \oddsidemargin 0.0 in \evensidemargin 0.0 in
\newenvironment{enumeratealpha}
{\begin{enumerate}[(a{\textup{)}}]}{\end{enumerate}}

\theoremstyle{definition}
\newtheorem{lem-rule}{Rule}
\newtheorem{thm}{Theorem}
\newtheorem{lem}{Lemma}[thm]
\newtheorem{prop}{Proposition}[thm]
\newtheorem{dfn}{Definitions}[thm]
% text macros
\def\cI{{\mathcal I}} \def\cR{{\mathcal R}} \def\cE{{\mathcal E}}
\def\cC{{\mathcal C}} \def\cF{{\mathcal F}} \def\cU{{\mathcal U}}
\def\cH{{\mathcal H}} \def\cD{{\mathcal D}} \def\cB{{\mathcal B}}
\def\cQ{{\mathcal Q}} \def\cV{{\mathcal V}} \def\cS{{\mathcal S}}
\def\cG{{\mathcal G}} \def\cA{{\mathcal A}}

\def\cId{{$\mathcal I$}} \def\cRd{{$\mathcal R$}} \def\cEd{{$\mathcal
    E$}} \def\cCd{{$\mathcal C$}} \def\cFd{{$\mathcal F$}}
\def\cUd{{$\mathcal U$}} \def\cHd{{$\mathcal H$}} \def\cDd{{$\mathcal
    D$}} \def\cBd{{$\mathcal B$}} \def\cQd{{$\mathcal Q$}}
\def\cVd{{$\mathcal V$}} \def\cSd{{$\mathcal S$}} \def\cGd{{$\mathcal
    G$}} \def\cAd{{$\mathcal A$}}

\bibliographystyle {IEEEtranS}
\begin {document}
\begin{thm}
  For a given network $N(S,T)$ with sensors $s\in S$ and targets $t\in T$, there exists an equivalent hypergraph $G(V,E)$ such that any solution to the target coverage problem in $N$ is a vertex cover in $G$, and any vertex cover of $G$ is a target cover for $N$.
\label{thm:equiv}
\end{thm}
\begin{proof}[Proof of Theorem~\ref{thm:equiv}]
\begin{prop}
For $N(S,T)$, define set of all sensors which can cover a target as $c(t_i)$ = \{$s \in S | $s$ covers $t$\}.  For example, if sensors $s_1, s_4, s_7$ are all of the sensors that cover $t_3$, $c(t_3) = \{s_1, s_4, s_5\}$. \label{prp:network}
\end{prop}
\begin{prop}To construct the equivalent hypergraph $G(V,E)$, as $V = S$ and $E$ is the set of hyper edges $c(t_i)$, for all $t_i \in T$. 
\label{prp:graph}
\end{prop}

To be shown: if $S' \subset S$ is a solution for target coverage in $N$, $f(S')$ is an edge cover of $G$. 

We proceed by direct proof. Let us define $S'\subset S$ such that $S'$ is a target cover of $N$. If $S'$ is a target cover, we know that 
\begin{equation}
\forall t \in T, \exists s \in t | s \in S'
\label{eqn:tcover}
\end{equation}
That is, for every target, there is a sensor in $S'$ that covers that target.

We also know that there exists an equivalent edge in $G$ for each target, defined as $g(t)$. We rewrite equation~\ref{eqn:tcover} to include proposition~\ref{prp:graph},
\begin{equation}
\forall t \in T, \exists s \in t | s \in S' \land f(s) \in g(t)
\label{eqn:extcover}
\end{equation}

Since the range of $g$ is comprehensive of $E$, this implies that for all edges,
\begin{equation}
\forall e \in E, \exists s \in S' | f(s) \in e
\end{equation}
Therefore, if we apply $f(s)$ to every $s \in S'$, we will create a vertex set $V'$ such that
\begin{equation}
\forall e \in E, \exists v \in V' | v \in e
\label{eqn:vcover}
\end{equation}
Equation~\ref{eqn:vcover} describes a vertex cover of $G$, which is what we wanted to find.

The second part of the theorem follows from the definition of bijection. 
\end{proof}

\end{document}