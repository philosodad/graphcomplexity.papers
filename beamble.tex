
% $Header: /Users/paul/Classes/8220/Presentation/RCS/route-recovery.tex,v 1.1 2009/04/17 07:03:56 paul Exp $

\documentclass{beamer}

% This file is a solution template for:

% - Giving a talk on some subject.
% - The talk is between 15min and 45min long.
% - Style is ornate.



% Copyright 2004 by Till Tantau <tantau@users.sourceforge.net>.
%
% In principle, this file can be redistributed and/or modified under
% the terms of the GNU Public License, version 2.
%
% However, this file is supposed to be a template to be modified
% for your own needs. For this reason, if you use this file as a
% template and not specifically distribute it as part of a another
% package/program, I grant the extra permission to freely copy and
% modify this file as you see fit and even to delete this copyright
% notice. 


\mode<presentation>
{
  \usetheme{Frankfurt}
  % or ...
  
  \setbeamercovered{transparent}
  % or whatever (possibly just delete it)

  \usefonttheme[onlymath]{serif}
  
}

\usepackage{verbatim}
\usepackage{multirow} \usepackage{enumerate}
\usepackage{amsmath,enumerate} \usepackage{amsthm}
%\usepackage{algcompatible}
%\usepackage{algpseudocode}
%\usepackage{algorithm}
%\usepackage{algorithmic}
\usepackage{pstricks}
\usepackage{amssymb, latexsym}
\usepackage{xfrac}
\usepackage{mathtools}
\usepackage{graphicx}
%\usepackage[captionskip=5pt, nearskip=5pt, font=small]{subfig}
\DeclareGraphicsRule{*}{mps}{*}{}
%\usepackage{listings}

%pgfsettings
%\usepackage{pgf}
%\usepackage{tikz}
%\usetikzlibrary{decorations.pathmorphing} % LATEX and plain TEX when using Tik Z
%\usetikzlibrary{positioning}
%\tikzstyle{vx}=[draw,circle,fill=black!50,minimum size=2pt, inner sep=0pt, node distance=15mm]
%\tikzstyle{bup}=[decoration={bent, aspect=.3, amplitude=4}, decorate, ->, >=stealth]
%\tikzstyle{bdn}=[decoration={bent, aspect=.3, amplitude=-4}, decorate, ->, >=stealth]
%\tikzstyle{BUP}=[decoration={bent, aspect=.3, amplitude=8}, decorate, ->, >=stealth]
%\tikzstyle{BDN}=[decoration={bent, aspect=.3, amplitude=-8}, decorate, ->, >=stealth]
\usepackage{pgf}
\usepackage{tikz}
\usetikzlibrary{decorations.pathmorphing} % LATEX and plain TEX when using Tik Z
\usetikzlibrary{positioning}
\usetikzlibrary{er}
\usetikzlibrary{automata}
\usetikzlibrary{shapes.geometric}
\tikzstyle{vx}=[draw,circle,fill=white,minimum size=2pt, inner sep=1pt, node distance=15mm]
\tikzstyle{ex}=[draw,rectangle,fill=white,minimum size=2pt, inner sep=3pt, node distance=15mm]
\tikzstyle{bup}=[semithick, decoration={bent, aspect=.3, amplitude=4}, decorate, ->, >=stealth]
\tikzstyle{bdn}=[semithick, decoration={bent, aspect=.3, amplitude=-4}, decorate, ->, >=stealth]
\tikzstyle{BUP}=[thick, decoration={bent, aspect=.3, amplitude=8}, decorate, ->, >=stealth]
\tikzstyle{BDN}=[thick, decoration={bent, aspect=.3, amplitude=-8}, decorate, ->, >=stealth]
\tikzstyle{MUP}=[thick, decoration={bent, aspect=.3, amplitude=16}, decorate, ->, >=stealth]
\tikzstyle{MDN}=[thick, decoration={bent, aspect=.3, amplitude=-16}, decorate, ->, >=stealth]
\tikzstyle{str}=[semithick, decorate, ->, >=stealth]
\tikzstyle{cr}=[draw, circle, fill=black!25,minimum size=150pt]

%styles for plots?
\tikzstyle{bls}=[blue, solid, mark=square*]
\tikzstyle{grt}=[red, solid, mark=*]
\tikzstyle{inv}=[draw=none]

\usepackage{algorithm, algorithmic}

\usepackage[english]{babel}
% or whatever

\usepackage[latin1]{inputenc}
% or whatever

\usepackage{times}
\usepackage[T1]{fontenc}
\usepackage{graphics}
% Or whatever. Note that the encoding and the font should match. If T1
% does not look nice, try deleting the line with the fontenc.

%pgfplots
\usepackage{pgfplots}
\usepackage{pgfplotstable}
\pgfplotstableread{plts/experiment8b1_av.tab}\averageone
\pgfplotstableread{plts/experiment8b2_av.tab}\averagetwo
\pgfplotstableread{plts/experiment8b3_av.tab}\averagethree
\pgfplotstableread{plts/experiment8b4_av.tab}\averagefour
\pgfplotstableread{plts/experiment9a_av.tab}\stepping
\pgfplotstableread{plts/experiment9a1_av.tab}\steppingone
\pgfplotstableread{plts/experiment9a2_av.tab}\steppingtwo
\pgfplotstableread{plts/experiment9a3_av.tab}\steppingthree
\pgfplotstableread{plts/experiment9a4_av.tab}\steppingfour
\pgfplotstableread{plts/experiment9b1_av.tab}\runningone
\pgfplotstableread{plts/experiment9b2_av.tab}\runningtwo
\pgfplotstableread{plts/experiment9b3_av.tab}\runningthree
\pgfplotstableread{plts/experiment9b4_av.tab}\runningfour
\pgfplotstableread{plts/experiment9b_av.tab}\running
\pgfplotstableread{plts/experiment9c_av.tab}\costcomp
\pgfplotstableread{plts/experiment9c1_av.tab}\costcompone
\pgfplotstableread{plts/experiment9c2_av.tab}\costcomptwo
\pgfplotstableread{plts/experiment9c3_av.tab}\costcompthree
\pgfplotstableread{plts/experiment8b1_rn.tab}\runsone
\pgfplotstableread{plts/experiment8b2_rn.tab}\runstwo
\pgfplotstableread{plts/experiment8b3_rn.tab}\runsthree
\pgfplotstableread{plts/experiment8b4_rn.tab}\runsfour
\pgfplotstableset{
  create on use/density/.style={
    create col/expr={\thisrow{nodes}+\thisrow{links}}}
    }
\pgfplotstableset{
  create on use/delta/.style={
    create col/expr={\thisrow{links}*2}}
    }
\pgfplotstableset{
  create on use/nodebylinks/.style={
    create col/expr={(\thisrow{nodes}*\thisrow{links})}}
    }
\pgfplotscreateplotcyclelist{three}{% 
  every mark/.append style={fill=teal}\\% 
  every mark/.append style={fill=green}\\% 
  every mark/.append style={fill=orange}\\% 
}
\pgfplotscreateplotcyclelist{four}{%
  every mark/.append style={fill=teal}\\%
  every mark/.append style={fill=green}\\%
  every mark/.append style={fill=orange}\\%
  every mark/.append style={fill=pink}\\%
}
\pgfplotscreateplotcyclelist{three-1-0}{%
  every mark/.append style={fill=teal}\\% 
  every mark/.append style={fill=green}\\% 
  every mark/.append style={fill=orange}\\%
	every mark/.append style={fill=none}\\% 
	every mark/.append style={fill=none}\\% 
	every mark/.append style={fill=none}\\% 
}
\pgfplotscreateplotcyclelist{three-0-1}{%
	every mark/.append style={fill=none}\\% 
	every mark/.append style={fill=none}\\% 
	every mark/.append style={fill=none}\\% 
  every mark/.append style={fill=teal}\\% 
  every mark/.append style={fill=green}\\% 
  every mark/.append style={fill=orange}\\%
}
\pgfplotscreateplotcyclelist{four-1-0}{%
  every mark/.append style={fill=teal}\\%
  every mark/.append style={fill=green}\\%
  every mark/.append style={fill=orange}\\%
  every mark/.append style={fill=pink}\\%
	every mark/.append style={fill=none}\\%
	every mark/.append style={fill=none}\\%
	every mark/.append style={fill=none}\\%
	every mark/.append style={fill=none}\\%
}
\pgfplotscreateplotcyclelist{four-0-1}{%
	every mark/.append style={fill=none}\\%
	every mark/.append style={fill=none}\\%
	every mark/.append style={fill=none}\\%
	every mark/.append style={fill=none}\\%
  every mark/.append style={fill=teal}\\%
  every mark/.append style={fill=green}\\%
  every mark/.append style={fill=orange}\\%
  every mark/.append style={fill=pink}\\%
}

%%some convenient text substitution
\def\vdp{Vertex-Disjoint Path}
\def\vdps{Vertex-Disjoint Paths}
\def\VDP{{\bf VDP}}
\def\VDPs{{\bf VDPs}}
\def\cI{{\mathcal I}} \def\cR{{\mathcal R}} \def\cE{{\mathcal E}}
\def\cC{{\mathcal C}} \def\cF{{\mathcal F}} \def\cU{{\mathcal U}}
\def\cH{{\mathcal H}} \def\cD{{\mathcal D}} \def\cB{{\mathcal B}}
\def\cQ{{\mathcal Q}} \def\cV{{\mathcal V}} \def\cS{{\mathcal S}}
\def\cG{{\mathcal G}} \def\cA{{\mathcal A}} \def\cO{{\mathcal O}}
\def\cW{{\mathcal W}} \def\cL{{\mathcal L}} 

\def\bI{{\mathbb I}} \def\bO{{\mathbb O}}
\def\bC{{\mathbb C}} \def\bM{{\mathbb M}}
\def\bId{{$\mathbb I$}} \def\bOd{{$\mathbb O$}}
\def\bCd{{$\mathbb C$}} \def\bMd{{$\mathbb M$}}

\def\cId{{$\mathcal I$}} \def\cRd{{$\mathcal R$}} \def\cEd{{$\mathcal E$}} 
\def\cCd{{$\mathcal C$}} \def\cFd{{$\mathcal F$}} \def\cUd{{$\mathcal U$}} 
\def\cHd{{$\mathcal H$}} \def\cDd{{$\mathcal D$}} \def\cBd{{$\mathcal B$}} 
\def\cQd{{$\mathcal Q$}} \def\cVd{{$\mathcal V$}} \def\cSd{{$\mathcal S$}} 
\def\cGd{{$\mathcal G$}} \def\cAd{{$\mathcal A$}} \def\cOd{{$\mathcal O$}}
\def\cWd{{$\mathcal W$}} \def\cLd{{$\mathcal L$}}

\def\suchthat{{\: |\:}}


% le bib style
\bibliographystyle {IEEEtranS}
