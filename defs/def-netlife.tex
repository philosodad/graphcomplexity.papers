\subsubsection{Network Lifetime}
\label{sub:net-life}
Given a sensor network $N(S,T)$, the network is covered when $\forall t \in T$, there is an active sensor $s \in S$ that is covering T. Because there is no constraint in the general case on the number of sensors covering $T$, the finding a sensor cover is equivalent to finding a vertex cover in a hypergraph. Assume each sensor has some battery lifetime. We add a constraint to the definition of a cover, which is that for a cover \bCd, the cover is valid if $\prod_{n \in \bC} w(n) > 0$, that is, as long as every sensor in the cover has positive battery life. Battery life for sensors in the cover decreases at a rate greater than that of sensors outside the cover, so over time, any given cover will have a sensor with battery life of zero, and the cover is no longer valid. The sensor network is alive as long as it is still possible to find a valid cover for the network. For a given network, $N$, and a time $t$, the network lifetime decision problem is whether $N$ can be kept alive for $t$. 
