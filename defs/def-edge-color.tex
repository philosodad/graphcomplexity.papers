Given an undirected Graph $G(V,E)$ and a set of colors $C$, an edge coloring of the graph is a mapping   
	\begin{align*} f: E \mapsto C \suchthat & \forall e_1(u,v), e_2(u,w) \in E, f(e_1) \ne f(e_2)
\end{align*} 

Informally, we would say that every edge in the graph is assigned a color, and no two incident edges in the graph are assigned the same color. The Minimum Edge Coloring Problem (MECP) is to find the smallest number of colors that can meet these constraints. This problem is known to be NP-complete.
