\documentclass[article, 10pt, letter]{IEEEtran}
\input{preamble.tex}
\bibliographystyle{IEEEtranS}

\begin{document}
\title{A General Purpose Automata for Distributed Graph Algorithms}

\author{\IEEEauthorblockN{J. Paul Daigle and Sushil K. Prasad}
\IEEEauthorblockA{Department of Computer Science\\
Georgia State University\\
Atlanta, Georgia 30303, USA\\}
}

\maketitle

\begin{abstract}
A consistent problem with distributed algorithms for graph problems is scalability. Here we present a scalable, persistently \(O(\Delta)\) automata that can easily be modified to solve a number of problems. We focus on the Minimum Weighted Vertex Cover problem, showing that our approach improves on the best known algorithm, reducing the number of communication rounds from \( O(\log n) \) to \( O(\log \Delta) \). We then show how our framework can easily be modified to solve for Edge Coloring.
\end{abstract}

\section{Introduction}


The Minimum Vertex Cover problem and its weighted variant are NP-Complete problems with several known linear time sequential algorithms that provide constant factor approximations. The existence of such algorithms suggests that it should be theoretically possible to discover a distributed or parallel algorithm to provide a constant factor approximation in constant time, however, it has been proven that no such algorithm exists\cite{1011811}. 

Here we present a distributed algorithm that provides a two-approximate solution to Minimum Weighted Vertex Cover in $O(\log\Delta)$ rounds with high probability, where $\Delta$ is the largest degree of the graph. This algorithm, based on Maximal Matchings, is the first distributed algorithm that we are aware of with a performance based on $\Delta$ rather than on the size of the graph. We extend our framework for this algorithm and show how it can be applied to two related problems, the Maximum Weighted Independent Set Problem and the Minimum Edge Coloring problem.

\subsection{Problem Definition}
\input{defs/vertex-cover-def.tex}
\subsection{Prior Work}

Sequential Linear time algorithms for covering problems are surveyed in detail in \cite{254190}. The seminal paper on Linear Programming techniques for constant ratio approximation of MWVC was published by Bar-Yehuda and Even in 1981 \cite{Bar-Yehuda:1981lr}. Gonzalez created a 2-optimal LP-Free linear time algorithm based on Maximal Matching in 1995 which is the basis of our distributed algorithm \cite{Gonzalez1995129}. 

We are aware of two distributed algorithms for minimum weighted vertex cover. A 2-optimal algorithm based on maximal matching \cite{1435381}, based on \cite{Israel:1986:FSR:5361.5365}, which uses a very similar approach to our own. This algorithm uses $O(\log n + \hat{W})$ communication rounds, where $\hat{W}$ is the average vertex weight. A simpler $O(\log n)$ algorithm is presented in \cite{1582746}, which relies on constructing disjoint trees.

\section{Algorithmic Framework}

Our framework is based on a simple handshake protocol, shown in Figure~\ref{fig:automata}. Each node begins a decision round in the \cCd, or {\em choose} state, and chooses randomly whether to transition to the \cId\ or \cLd\ state. Nodes in the \cId\ state choose a single neighbor and issue an {\em invitation} to that neighbor to form a pair during this round. And invitation should contain all of the information that the invitee needs to compute the sub-problem for this round. Nodes in the \cLd\ state collect their invitations. In the next step, these position are reversed, each node that was in the \cLd\ state transitions to the \cRd\ state, chooses a single invitation from their collection and responds to that invitation. Nodes in the \cId\ state transition to \cWd\ state filter the local messages, looking for a response to the prior invitation. Responses, like invitations, should contain all the data needed to solve the sub-problem for this round. Finally, all nodes transition to the \cUd\ state and solve their local-subproblem, broadcasting the results in the \cEd state. Nodes with no outstanding problems transition to \cDd, nodes with outstanding sub-problems go back to \cCd.

\input{figs/fig-automata.tex}

\subsection{Application to Minimum Weighted Vertex Cover}
\label{sub:algorithms-dgmm}
\subsubsection{Description}
\label{ssb:algorithms-dgmm-description}

Algorithm~\ref{alg:dgmm} is our distributed implementation of the 2-optimal minimum weighted vertex cover algorithm presented by Gonzalez \cite{Gonzalez1995129}. The Gonzalez algorithm, Generalized Maximal Matching for MWVC (or GMM), proceeds by selecting each edge in turn and choosing one of the endpoints of that edge to add to the cover. The sequential algorithm goes through each edge in turn and assigns the edge a weight according to~\eqref{eqn:gmm}.

\input{eqns/eqn-gmm.tex}

So if there are no previously weighted edges incident to either endpoint, the weight of the edge is $min(w(u),w(v))$. A vertex $u$ joins the cover when, for all it's weighted edges, $\sum_i w(e(u,i)) \equiv w(u)$. When ~\eqref{eqn:gmm} is applied to subsequent incident edges of $u$, the result will be 0. The algorithm terminates when each edge has been weighted. 

In GMM, every edge is examined exactly one time. If no endpoints of the edge are in the cover, one endpoint will join the cover. Finally, all the edges in a matching can be evaluated in arbitrary order without side effects. We explore this third point further in Section~\ref{ssb:algorithms-dgmm-performance}. 

The distributed version of the algorithm chooses some disjoint set of edges and assigns weights to those edges according to~\eqref{eqn:gmm}. The framework automata is used to construct the set of disjoint edges. Each invitation contains the current incident weight\footnote{The sum of the edge weights} of the inviting node, and each response contains the incident weight of the responding node. Therefore, during the \cUd\ state, vertex pairs formed during the invitation/acceptance phases are able to assign a weight to the edge between them independently using equation~\ref{eqn:gmm}, which allows each node in the pair to decide whether or not to join the cover. During the \cEd\ state there is some update of unweighted edges, as each node can assign a weight of zero to any edges incident to a vertex which has joined the cover.

\input{alg/alg-dgmm.tex}

\subsection{Performance}

\label{ssb:algorithms-dgmm-performance}

Despite its overall simplicity, Algorithm~\ref{alg:dgmm} has a running time of $O(\log \Delta)$, which improves on the previous best result of $O(\log n)$ in \cite{1582746} without sacrificing the performance boundary of two-optimality. A formal proof of our performance claims follows.

\input{proofs/prf-dgmm-term-log.tex}

\subsection{Application to other problems}

Any problem that can be solved by computations over matchings can be solved in a distributed fashion using this framework approach. In this section we briefly examine two such problems, the Independent Set problem and the Edge Coloring problem.

\subsubsection{Independent Set Problem}
\paragraph{Definition}
\input{defs/def-is.tex}
\paragraph{Application of the Framework}

The application of the framework to the Maximum Weighted Independent Set problem follows trivially from Algorithm~\ref{alg:dgmm}. These are the only modifications needed:

\begin{enumerate}
\item Instead of choosing to join the cover, minimum nodes choose to exclude themselves from the set.
\item Nodes join the set when all of their neighbors have chosen to exclude themselves from the set.\label{en:modlist-lastline}
\end{enumerate}

If we examine Algorithm~\ref{alg:dgmm}, the MWIS problem can be solved with this automata by replacing all incidences of $True$ with $False$ and vice-versa. While this implies that the number of communication rounds for MWIS would be $O(\log \Delta)$ as well, a formal proof of the quality of this approach is outside the scope of this paper.

\subsubsection{Edge Coloring}
\paragraph{Definition}
Given an undirected Graph $G(V,E)$ and a set of colors $C$, an edge coloring of the graph is a mapping   
	\begin{align*} f: E \mapsto C \suchthat & \forall e_1(u,v), e_2(u,w) \in E, f(e_1) \ne f(e_2)
\end{align*} 

Informally, we would say that every edge in the graph is assigned a color, and no two incident edges in the graph are assigned the same color. The Minimum Edge Coloring Problem (MECP) is to find the smallest number of colors that can meet these constraints. This problem is known to be NP-complete.

\paragraph{Application of the Framework}
\label{par:apply-edge-color}

Algorithm~\ref{alg:edge-color} shows the application of the framework to the Edge Coloring problem. The primary changes concern the maintenance of the open color lists and the content of broadcast messages. Of particular concern is the maintenance of usable color lists--the requires the maintenance of a dictionary, initiliazed in Line~\algref{alg:edge-color}{alglin:ec-init-dict}, which maintains a separate list of used colors for each neighbor.


\begin{algorithm}
\caption{Distributed Maximal-Matching Based Edge-Coloring Algorithm}
\begin{algorithmic}[1]
%\Require {$G(V,E)$: a Communicaton Network}
\ForAll {$v_u \in V$ in parrallel}
\State {$live_u \gets C$} \Comment All colors are available \label{alglin:ec-init-color}
\State {$dlist_u \gets \{\}$} \Comment No colors are used \label{alglin:ec-init-dict}
\State {$clist_u \gets [\,]$} \Comment No colors are assigned
\State state $\leftarrow$ \cCd
\Repeat
\If {state = \cCd}
\State {State $\gets (\cI \lor \cL)$} \Comment Coin toss selects state

\ElsIf {state = \cId}
\State {Randomly select an uncolored edge, $e_{u,v}$}\label{alglin:ec-issue-invite}
\State {$c \gets live_u(1)$} \Comment assign first available color to $c$
\State {Broadcast $I_u^v,c$} \Comment $u$ Invites $v$ to color $e_{u,v} \text{ with } c$ 
\State {state $\leftarrow$ \cWd}

\ElsIf {state = \cLd}
\State {Recieve $I_x^y,c$} \Comment all local invites
\If {$y = u$} \Comment invite is targeted to $v_u$
\State {store $I_x^y,c$}
\EndIf
\State {State $\gets \cR$}

\ElsIf {state = \cRd}
\State {Randomly Select $I_v^u,c$} \Comment from stored invites \label{alglin:ec-choose-invite}
\State {Broadcast $R_u^v,c $} \Comment $u$ accepts $v's$ invitation
\State {Assign $c \mapsto e_{u,v}$} 
\State {$clist_u \hookleftarrow c$} \Comment Append c to assigned colors
\State {state $\gets \cC$}

\ElsIf {state = \cWd}
\State {Recieve $R_x^y,c$} \Comment all local responses
\If {$y = u$} \Comment response is to $v_u$
\State {Assign $c \mapsto e_{u,v}$} 
\State {$clist \hookleftarrow c$} 
\EndIf

\ElsIf {state = \cUd}
\State {Broadcast $clist_u$} \Comment Broadcast all assigned edge colors
\State {Recieve $clist_v$} \Comment Receive neighbors assigned colors
\State {$dlist_u \hookleftarrow clist_v$} 

\ElsIf {State = \cEd}
\State {Subtract $clist_u$ from $live_u$} \Comment update usable colors
\EndIf
\Until {All edges are assigned a color}\label{alglin:ec-end-while}
\EndFor
\end{algorithmic}
\label{alg:edge-color}
\end{algorithm}



Algorithm~\ref{alg:edge-color} uses a simple greedy approach. Each node keeps track of what colors each of it's neighbors have used (shared during the \cEd stage). Colors are taken from an ordered list which is part of the initialization process (Line~\algref{alg:edge-color}{alglin:ec-init-color}). When sending an invitation to a neighbor to set the color for an edge, the node chooses the lowest indexed color that doesn't appear either on its own list of used colors or the neighbor's list of used colors.

Intuitively, we can see that the algorithm will be correct from the process. We know from Section~\ref{ssb:algorithms-dgmm-performance} that no two contiguous edges can be decided in a single round, so in order for two contiguous edges, $e(u,v), e(u,w)$ to have the same color, node $u$ would have to invite a neighbor to share a color that $u$ is already using. We have also shown the automata will not terminate before it reaches it's end condition, so as long as two nodes share an uncolored edge, they will continue to attempt to form a partnership and color the edge. 

A formal proof of Algorithm~\ref{alg:edge-color} is beyond the scope of this paper, however, in limited experiments, we had a running time of $O{\Delta}$ with quality between $\Delta$ and $\Delta+2$. This is detailed in Section~{sec:experiment}.


\section{Experiment and Results}

In order to confirm our performance predictions, we designed a simulator to implement our algorithm (DGMM) and the algorithm of Koufoganiakkis and Young (K/Y). We generated random weighted graphs for $n=120, n=240, n=480, \text{and} n=960$, with average degrees of 3, 6, 12, 24, 48 and 96. 50 graphs were generated for each combination of sizes and average degree, for a total of 1200 graphs. DGMM and K/Y were used to produce a vertex cover for each graph, and the total weight and number of communication rounds were tallied and averaged for each of the 24 graph sizes. Our results showed that DGMM produced covers of approximately the same weight as K/Y in significantly fewer communication rounds in each case. Figures~\ref{plt:dgmm-comp} and \ref{plt:mwvc-rn} show our experimental results.

\input{figs/fig-plt-mwvc-rn.tex}
\input{figs/fig-plt-dgmm-comp.tex}

In Figure~\ref{plt:dgmm-comp}, the results for both algorithms are nearly identical, indicating that both algorithms found essentially the same covers for the input graphs. In Figure~\ref{plt:mwvc-rn}, the lower 4 lines indicate the number of rounds required to resolve DGMM, and the upper four the number of rounds required to resolve K/Y. The logarithmic growth confirms our experimental predictions, while the minor differences between different input sizes are accounted for by the variation in degree between larger and smaller random graphs.

In the case of the Edge Color Algorithm, we only ran a small set of experiments, 10 experiments for with an no of 200 or 400 and average degrees of 8 and 24. Our small sample confirmed our expectation of $O(\Delta)$ running time, and in no individual case did the number of colors used exceed $\Delta + 2$.

\bibliography{vertex_bib}
\end{document}
