\documentclass[review ]{elsarticle}
\input{preamble.tex}
\bibliographystyle {elsarticle-num}
\begin{document}
\title{A Simple Distributed Algorithm for Minimum Weighted Vertex Cover}
\author[gsu]{J. Paul Daigle}
\ead{j.paul.daigle@gmail.com}
\author[gsu]{Sushil K. Prasad}
\ead{sushil.prasad@gmail.com}
\address[gsu]{Department of Computer Science, Georgia State University, P.O. Box 3994, Atlanta, GA  30302-3994}
\begin{abstract} Vertex cover, a minimal set of nodes to cover all edges in a graph, is an abstraction of coverage problems in sensor networks, transportation networks, etc., and is a well-known NP-hard problem.  Minimum weighted vertex cover (MWVC) problem asks for further minimizing the cumulative weight of a vertex cover.  We present a new distributed 1-hop algorithm for the MWVC problem with theoretical and practical value.  Our first 1-hop approximation algorithm, based on matching a maximal set of non-adjacent edges, is provably 2-optimal with a communication complexity of $O(\log\Delta)$.   It compares very well with the current state-of-art in quality while significantly reducing communication cost.
\end{abstract}

\begin{keyword}Distributed Algorithms \sep Graph Algorithms \sep Vertex Cover \end{keyword}
\maketitle

\section{Introduction}

The Minimum Vertex Cover problem and its weighted variant are NP-Complete problems with several known linear time sequential algorithms that provide constant factor approximations. The existence of such algorithms suggests that it should be theoretically possible to discover a distributed or parallel algorithm to provide a constant factor approximation in constant time, however, it has been proven that no such algorithm exists\cite{1011811}. 

Here we present a distributed algorithm that provides a two-approximate solution to Minimum Weighted Vertex Cover in $O(\log\Delta)$ rounds with high probability, where $\Delta$ is the largest degree of the graph. This algorithm, based on Maximal Matchings, is the first distributed algorithm that we are aware of with a performance based on $\Delta$ rather than on the size of the graph. 

\subsection{Problem Definition}
\input{defs/vertex-cover-def.tex}
\subsection{Prior Work}

Sequential Linear time algorithms for covering problems are surveyed in detail in \cite{254190}. The seminal paper on Linear Programming techniques for constant ratio approximation of MWVC was published by Bar-Yehuda and Even in 1981 \cite{Bar-Yehuda:1981lr}. Gonzalez created a 2-optimal LP-Free linear time algorithm based on Maximal Matching in 1995 which is the basis of our distributed algorithm \cite{Gonzalez1995129}. 

We are aware of two distributed algorithms for minimum weighted vertex cover. A 2-optimal algorithm based on maximal matching \cite{1435381}, based on \cite{Israel:1986:FSR:5361.5365}, which uses a very similar approach to our own. This algorithm uses $O(\log n + \hat{W})$ communication rounds, where $\hat{W}$ is the average vertex weight. A simpler $O(\log n)$ algorithm is presented in \cite{1582746}, which relies on constructing disjoint trees.

\section{Algorithm}
\subsection{Description}
\input{secs/dgmm-descr.tex}
\subsection{Performance}
\label{ssb:algorithms-dgmm-performance}
\input{proofs/prf-dgmm-term-log.tex}
\section{Experiment and Results}

In order to confirm our performance predictions, we designed a simulator to implement our algorithm (DGMM) and the algorithm of Koufoganiakkis and Young (K/Y). We generated random weighted graphs for $n=120, n=240, n=480, \text{and} n=960$, with average degrees of 3, 6, 12, 24, 48 and 96. 50 graphs were generated for each combination of sizes and average degree, for a total of 1200 graphs. DGMM and K/Y were used to produce a vertex cover for each graph, and the total weight and number of communication rounds were tallied and averaged for each of the 24 graph sizes. Our results showed that DGMM produced covers of approximately the same weight as K/Y in significantly fewer communication rounds in each case. Figures~\ref{plt:dgmm-comp} and \ref{plt:mwvc-rn} show our experimental results.

\begin{figure}[htp]
	\subfloat[Comparison of Weights]{\input{plts/plt-dgmm-comp.tex} \label{plt:dgmm-comp}}
	\subfloat[Comparison of Communication Rounds]{\input{plts/plt-mwvc-rn.tex} \label{plt:mwvc-rn}}
\end{figure}


\section{Discussion}

\subsection{Discussion}
The general approach of this algorithm, that of finding a matching on the graph, can be extended to solve other problems. For instance, it is easy to see how a matching could be used to generate an edge coloring or an independent set. In practical applications, such as in sensor networks, reducing the number of communication rounds necessary to solve covering or coloring problems can have positive results for the lifetime of the network. Algorithms that are applied to such networks must also be scalable. The performance gains for our algorithm are considerable in large graphs, particularly sparse graphs. In applications that require large unit disk graphs, we expect our approach to yield practical results.

\subsection{Conclusion}

We have introduced a new distributed algorithm for the Minimum Weighted Vertex Cover based on maximal matchings. The previous lower bound for this problem was $O(\log n)$ with high probability, we have shown that our algorithm improves this to $O(\log \Delta)$ for random graphs. 

\bibliography{vertex_bib}
\end{document}

