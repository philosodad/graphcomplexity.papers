\documentclass[conference, 11pt]{IEEEtran} 
\usepackage{verbatim}
\usepackage{multirow} \usepackage{enumerate}
\usepackage{amsmath,enumerate} \usepackage{amsthm}
\usepackage{algorithm}
\usepackage{algorithmic}

\usepackage{graphicx}
\DeclareGraphicsRule{*}{mps}{*}{}

\usepackage{pgf}
\usepackage{tikz}
\usetikzlibrary{decorations.pathmorphing} % LATEX and plain TEX when using Tik Z
\usetikzlibrary{positioning}
\usetikzlibrary{er}
\tikzstyle{vx}=[draw,circle,fill=black!50,minimum size=2pt, inner sep=1pt, node distance=15mm]
\tikzstyle{bup}=[semithick, decoration={bent, aspect=.3, amplitude=4}, decorate, ->, >=stealth]
\tikzstyle{bdn}=[semithick, decoration={bent, aspect=.3, amplitude=-4}, decorate, ->, >=stealth]
\tikzstyle{BUP}=[thick, decoration={bent, aspect=.3, amplitude=8}, decorate, ->, >=stealth]
\tikzstyle{BDN}=[thick, decoration={bent, aspect=.3, amplitude=-8}, decorate, ->, >=stealth]
\tikzstyle{MUP}=[thick, decoration={bent, aspect=.3, amplitude=16}, decorate, ->, >=stealth]
\tikzstyle{MDN}=[thick, decoration={bent, aspect=.3, amplitude=-16}, decorate, ->, >=stealth]
\tikzstyle{str}=[semithick, decorate, ->, >=stealth]
\tikzstyle{cr}=[draw, circle, fill=black!25,minimum size=150pt]

% \paperheight=11in \paperwidth=8.5in \textheight=9.0in
% \textwidth=6.5in \voffset=-.875in \hoffset=-.875in
\newenvironment{code} {\begin {quote}\begin{footnotesize}}
    {\end{footnotesize}\end{quote}}

% \oddsidemargin 0.0 in \evensidemargin 0.0 in
\newenvironment{enumeratealpha}
{\begin{enumerate}[(a{\textup{)}}]}{\end{enumerate}}

\theoremstyle{definition}
\newtheorem{lem-rule}{Rule}

% text macros
\def\cI{{\mathcal I}} \def\cR{{\mathcal R}} \def\cE{{\mathcal E}}
\def\cC{{\mathcal C}} \def\cF{{\mathcal F}} \def\cU{{\mathcal U}}
\def\cH{{\mathcal H}} \def\cD{{\mathcal D}} \def\cB{{\mathcal B}}
\def\cQ{{\mathcal Q}} \def\cV{{\mathcal V}} \def\cS{{\mathcal S}}
\def\cG{{\mathcal G}} \def\cA{{\mathcal A}}

\def\cId{{$\mathcal I$}} \def\cRd{{$\mathcal R$}} \def\cEd{{$\mathcal
    E$}} \def\cCd{{$\mathcal C$}} \def\cFd{{$\mathcal F$}}
\def\cUd{{$\mathcal U$}} \def\cHd{{$\mathcal H$}} \def\cDd{{$\mathcal
    D$}} \def\cBd{{$\mathcal B$}} \def\cQd{{$\mathcal Q$}}
\def\cVd{{$\mathcal V$}} \def\cSd{{$\mathcal S$}} \def\cGd{{$\mathcal
    G$}} \def\cAd{{$\mathcal A$}}

\bibliographystyle {IEEEtranS}
\begin {document}

\title{IEEE Paper} 

\author{\IEEEauthorblockN{J. Paul Daigle}
\IEEEauthorblockA{Department of Computer Science\\
Georgia State University\\
Atlanta, Georgia 30303\\
Email: jdaigle1@student.gsu.edu}
}

\maketitle

\begin{abstract}

\end{abstract}

\section{Results} 
In this section, we evaluate the performance of our algorithm in two cases. First, we compare our results to a baseline 2-approximation algorithm for unweighted vertex cover.\cite{500824} Next, we evaluate the performance of our algorithm against the distributed combinatorial 2-approximation algorithm for weighted vertex cover presented by Koufogiannakis and Young.\cite{1582746} The simulations were programmed using Python.\footnote{The code for this project is available on google code as open source software, this paper references changeset 6fa22bd4b8d9.} To adapt the framework based target coverage algorithm presented by Dhawan and Prasad\cite{IPDPS.2008.45361}, edges between nodes are represented as targets. 
\subsection{Unweighted Cover}
To calculate the unweighted cover, a graph containing 100 unweighted nodes was assigned a random number of edges in logarithmic steps between 100 and 4500 (forming a clique). 20 graphs were generated in each size. Vertex covers were calculated separately using the framework algorithm versus the standard reduction algorithm. The framework produced covers that were consitent with a constant 2-$\epsilon$ approximation. Figure~\ref{fig:unweight_covers} shows the average performance of both algorithms with the total degree of the tested graphs compared to the size of the vertex cover. Sensors were deployed in a square area and assumed to start with some small variation in battery life. Communication rounds were not modeled. The 2-approximation algorithm is known to have a communications cost of $O(log p)$ and the framework algorithm is bounded by $O(\Delta^\Delta)$, which can be viewed as a constant for sparse graphs.

\begin{figure}
  \begin{tikzpicture}[x=.05cm, y=.1 cm, only marks]
    \draw plot[mark=*, xshift=2pt, only marks] file {unweighted_fra.dat};
    \draw plot[mark=+, xshift=2pt, only marks] file {unweighted_seq.dat};
  \end{tikzpicture}
\end{figure}

  
\bibliography{vertex_bib}
\end {document}
