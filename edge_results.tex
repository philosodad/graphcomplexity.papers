\documentclass[conference, 11pt]{IEEEtran} 
\usepackage{verbatim}
\usepackage{multirow} \usepackage{enumerate}
\usepackage{amsmath,enumerate} \usepackage{amsthm}
\usepackage{algorithm}
\usepackage{algorithmic}
\usepackage{pstricks}
\usepackage{amssymb, latexsym}
\usepackage{graphicx}
\DeclareGraphicsRule{*}{mps}{*}{}

\usepackage{pgf}
\usepackage{tikz}
\usetikzlibrary{decorations.pathmorphing} % LATEX and plain TEX when using Tik Z
\usetikzlibrary{positioning}
\usetikzlibrary{er}
\tikzstyle{vx}=[draw,circle,fill=black!50,minimum size=2pt, inner sep=1pt, node distance=15mm]
\tikzstyle{bup}=[semithick, decoration={bent, aspect=.3, amplitude=4}, decorate, ->, >=stealth]
\tikzstyle{bdn}=[semithick, decoration={bent, aspect=.3, amplitude=-4}, decorate, ->, >=stealth]
\tikzstyle{BUP}=[thick, decoration={bent, aspect=.3, amplitude=8}, decorate, ->, >=stealth]
\tikzstyle{BDN}=[thick, decoration={bent, aspect=.3, amplitude=-8}, decorate, ->, >=stealth]
\tikzstyle{MUP}=[thick, decoration={bent, aspect=.3, amplitude=16}, decorate, ->, >=stealth]
\tikzstyle{MDN}=[thick, decoration={bent, aspect=.3, amplitude=-16}, decorate, ->, >=stealth]
\tikzstyle{str}=[semithick, decorate, ->, >=stealth]
\tikzstyle{cr}=[draw, circle, fill=black!25,minimum size=150pt]

% \paperheight=11in \paperwidth=8.5in \textheight=9.0in
% \textwidth=6.5in \voffset=-.875in \hoffset=-.875in
\newenvironment{code} {\begin {quote}\begin{footnotesize}}
    {\end{footnotesize}\end{quote}}

% \oddsidemargin 0.0 in \evensidemargin 0.0 in
\newenvironment{enumeratealpha}
{\begin{enumerate}[(a{\textup{)}}]}{\end{enumerate}}

\theoremstyle{definition}
\newtheorem{lem-rule}{Rule}

% text macros
\def\cI{{\mathcal I}} \def\cR{{\mathcal R}} \def\cE{{\mathcal E}}
\def\cC{{\mathcal C}} \def\cF{{\mathcal F}} \def\cU{{\mathcal U}}
\def\cH{{\mathcal H}} \def\cD{{\mathcal D}} \def\cB{{\mathcal B}}
\def\cQ{{\mathcal Q}} \def\cV{{\mathcal V}} \def\cS{{\mathcal S}}
\def\cG{{\mathcal G}} \def\cA{{\mathcal A}}

\def\cId{{$\mathcal I$}} \def\cRd{{$\mathcal R$}} \def\cEd{{$\mathcal
    E$}} \def\cCd{{$\mathcal C$}} \def\cFd{{$\mathcal F$}}
\def\cUd{{$\mathcal U$}} \def\cHd{{$\mathcal H$}} \def\cDd{{$\mathcal
    D$}} \def\cBd{{$\mathcal B$}} \def\cQd{{$\mathcal Q$}}
\def\cVd{{$\mathcal V$}} \def\cSd{{$\mathcal S$}} \def\cGd{{$\mathcal
    G$}} \def\cAd{{$\mathcal A$}}

\bibliographystyle {IEEEtranS}
\begin {document}

\title{IEEE Paper} 

\author{\IEEEauthorblockN{J. Paul Daigle}
\IEEEauthorblockA{Department of Computer Science\\
Georgia State University\\
Atlanta, Georgia 30303\\
Email: jdaigle1@student.gsu.edu}
}

\maketitle

\begin{abstract}

\end{abstract}

\section{Results} 
In this section, we evaluate the performance of our algorithm in two cases. First, we compare our results to a baseline 2-approximation algorithm for unweighted vertex cover.\cite{500824} Next, we evaluate the performance of our algorithm against the distributed combinatorial 2-approximation algorithm for weighted vertex cover presented by Koufogiannakis and Young.\cite{1582746} The simulations were programmed using Python.\footnote{The code for this project is available on google code as open source software, this paper references changeset 6fa22bd4b8d9.} To adapt the framework based target coverage algorithm presented by Dhawan and Prasad\cite{IPDPS.2008.45361}, edges between nodes are represented as targets. 
\subsection{Unweighted Cover}
To calculate the unweighted cover, a graph containing 100 unweighted nodes was assigned a random number of edges in logarithmic steps between 100 and 4500 (forming a clique). 20 graphs were generated in each size. Vertex covers were calculated separately using the framework algorithm versus the standard reduction algorithm. The framework produced covers that were consitent with a constant 2-$\epsilon$ approximation. Figure~\ref{fig:unweight_covers} shows the average performance of both algorithms with the total degree of the tested graphs compared to the size of the vertex cover. Sensors were deployed in a square area and assumed to start with some small variation in battery life. Communication rounds were not modeled. The 2-approximation algorithm is known to have a communications cost of $O(log p)$ and the framework algorithm is bounded by $O(\Delta^\Delta)$, which can be viewed as a constant for sparse graphs.

\begin{figure}[width=2in]
  \label{fig:unweight_covers}
  \caption{Unweighted Case Versus 2-approximation}
  % GNUPLOT: LaTeX picture
\setlength{\unitlength}{0.240900pt}
\ifx\plotpoint\undefined\newsavebox{\plotpoint}\fi
\begin{picture}(1050,630)(0,0)
\sbox{\plotpoint}{\rule[-0.200pt]{0.400pt}{0.400pt}}%
\put(191.0,131.0){\rule[-0.200pt]{4.818pt}{0.400pt}}
\put(171,131){\makebox(0,0)[r]{ 25}}
\put(980.0,131.0){\rule[-0.200pt]{4.818pt}{0.400pt}}
\put(191.0,223.0){\rule[-0.200pt]{4.818pt}{0.400pt}}
\put(171,223){\makebox(0,0)[r]{ 30}}
\put(980.0,223.0){\rule[-0.200pt]{4.818pt}{0.400pt}}
\put(191.0,315.0){\rule[-0.200pt]{4.818pt}{0.400pt}}
\put(171,315){\makebox(0,0)[r]{ 35}}
\put(980.0,315.0){\rule[-0.200pt]{4.818pt}{0.400pt}}
\put(191.0,406.0){\rule[-0.200pt]{4.818pt}{0.400pt}}
\put(171,406){\makebox(0,0)[r]{ 40}}
\put(980.0,406.0){\rule[-0.200pt]{4.818pt}{0.400pt}}
\put(191.0,498.0){\rule[-0.200pt]{4.818pt}{0.400pt}}
\put(171,498){\makebox(0,0)[r]{ 45}}
\put(980.0,498.0){\rule[-0.200pt]{4.818pt}{0.400pt}}
\put(191.0,590.0){\rule[-0.200pt]{4.818pt}{0.400pt}}
\put(171,590){\makebox(0,0)[r]{ 50}}
\put(980.0,590.0){\rule[-0.200pt]{4.818pt}{0.400pt}}
\put(191.0,131.0){\rule[-0.200pt]{0.400pt}{4.818pt}}
\put(191,90){\makebox(0,0){50}}
\put(191.0,570.0){\rule[-0.200pt]{0.400pt}{4.818pt}}
\put(219.0,131.0){\rule[-0.200pt]{0.400pt}{4.818pt}}
\put(219,90){\makebox(0,0){51}}
\put(219.0,570.0){\rule[-0.200pt]{0.400pt}{4.818pt}}
\put(247.0,131.0){\rule[-0.200pt]{0.400pt}{4.818pt}}
\put(247,90){\makebox(0,0){52}}
\put(247.0,570.0){\rule[-0.200pt]{0.400pt}{4.818pt}}
\put(275.0,131.0){\rule[-0.200pt]{0.400pt}{4.818pt}}
\put(275,90){\makebox(0,0){53}}
\put(275.0,570.0){\rule[-0.200pt]{0.400pt}{4.818pt}}
\put(303.0,131.0){\rule[-0.200pt]{0.400pt}{4.818pt}}
\put(303,90){\makebox(0,0){54}}
\put(303.0,570.0){\rule[-0.200pt]{0.400pt}{4.818pt}}
\put(330.0,131.0){\rule[-0.200pt]{0.400pt}{4.818pt}}
\put(330,90){\makebox(0,0){55}}
\put(330.0,570.0){\rule[-0.200pt]{0.400pt}{4.818pt}}
\put(358.0,131.0){\rule[-0.200pt]{0.400pt}{4.818pt}}
\put(358,90){\makebox(0,0){97}}
\put(358.0,570.0){\rule[-0.200pt]{0.400pt}{4.818pt}}
\put(386.0,131.0){\rule[-0.200pt]{0.400pt}{4.818pt}}
\put(386,90){\makebox(0,0){98}}
\put(386.0,570.0){\rule[-0.200pt]{0.400pt}{4.818pt}}
\put(414.0,131.0){\rule[-0.200pt]{0.400pt}{4.818pt}}
\put(414,90){\makebox(0,0){140}}
\put(414.0,570.0){\rule[-0.200pt]{0.400pt}{4.818pt}}
\put(442.0,131.0){\rule[-0.200pt]{0.400pt}{4.818pt}}
\put(442,90){\makebox(0,0){141}}
\put(442.0,570.0){\rule[-0.200pt]{0.400pt}{4.818pt}}
\put(470.0,131.0){\rule[-0.200pt]{0.400pt}{4.818pt}}
\put(470,90){\makebox(0,0){183}}
\put(470.0,570.0){\rule[-0.200pt]{0.400pt}{4.818pt}}
\put(498.0,131.0){\rule[-0.200pt]{0.400pt}{4.818pt}}
\put(498,90){\makebox(0,0){225}}
\put(498.0,570.0){\rule[-0.200pt]{0.400pt}{4.818pt}}
\put(526.0,131.0){\rule[-0.200pt]{0.400pt}{4.818pt}}
\put(526,90){\makebox(0,0){226}}
\put(526.0,570.0){\rule[-0.200pt]{0.400pt}{4.818pt}}
\put(554.0,131.0){\rule[-0.200pt]{0.400pt}{4.818pt}}
\put(554,90){\makebox(0,0){268}}
\put(554.0,570.0){\rule[-0.200pt]{0.400pt}{4.818pt}}
\put(582.0,131.0){\rule[-0.200pt]{0.400pt}{4.818pt}}
\put(582,90){\makebox(0,0){310}}
\put(582.0,570.0){\rule[-0.200pt]{0.400pt}{4.818pt}}
\put(609.0,131.0){\rule[-0.200pt]{0.400pt}{4.818pt}}
\put(609,90){\makebox(0,0){352}}
\put(609.0,570.0){\rule[-0.200pt]{0.400pt}{4.818pt}}
\put(637.0,131.0){\rule[-0.200pt]{0.400pt}{4.818pt}}
\put(637,90){\makebox(0,0){394}}
\put(637.0,570.0){\rule[-0.200pt]{0.400pt}{4.818pt}}
\put(665.0,131.0){\rule[-0.200pt]{0.400pt}{4.818pt}}
\put(665,90){\makebox(0,0){436}}
\put(665.0,570.0){\rule[-0.200pt]{0.400pt}{4.818pt}}
\put(693.0,131.0){\rule[-0.200pt]{0.400pt}{4.818pt}}
\put(693,90){\makebox(0,0){478}}
\put(693.0,570.0){\rule[-0.200pt]{0.400pt}{4.818pt}}
\put(721.0,131.0){\rule[-0.200pt]{0.400pt}{4.818pt}}
\put(721,90){\makebox(0,0){561}}
\put(721.0,570.0){\rule[-0.200pt]{0.400pt}{4.818pt}}
\put(749.0,131.0){\rule[-0.200pt]{0.400pt}{4.818pt}}
\put(749,90){\makebox(0,0){603}}
\put(749.0,570.0){\rule[-0.200pt]{0.400pt}{4.818pt}}
\put(777.0,131.0){\rule[-0.200pt]{0.400pt}{4.818pt}}
\put(777,90){\makebox(0,0){645}}
\put(777.0,570.0){\rule[-0.200pt]{0.400pt}{4.818pt}}
\put(805.0,131.0){\rule[-0.200pt]{0.400pt}{4.818pt}}
\put(805,90){\makebox(0,0){728}}
\put(805.0,570.0){\rule[-0.200pt]{0.400pt}{4.818pt}}
\put(833.0,131.0){\rule[-0.200pt]{0.400pt}{4.818pt}}
\put(833,90){\makebox(0,0){770}}
\put(833.0,570.0){\rule[-0.200pt]{0.400pt}{4.818pt}}
\put(861.0,131.0){\rule[-0.200pt]{0.400pt}{4.818pt}}
\put(861,90){\makebox(0,0){853}}
\put(861.0,570.0){\rule[-0.200pt]{0.400pt}{4.818pt}}
\put(888.0,131.0){\rule[-0.200pt]{0.400pt}{4.818pt}}
\put(888,90){\makebox(0,0){895}}
\put(888.0,570.0){\rule[-0.200pt]{0.400pt}{4.818pt}}
\put(916.0,131.0){\rule[-0.200pt]{0.400pt}{4.818pt}}
\put(916,90){\makebox(0,0){978}}
\put(916.0,570.0){\rule[-0.200pt]{0.400pt}{4.818pt}}
\put(944.0,131.0){\rule[-0.200pt]{0.400pt}{4.818pt}}
\put(944,90){\makebox(0,0){1061}}
\put(944.0,570.0){\rule[-0.200pt]{0.400pt}{4.818pt}}
\put(972.0,131.0){\rule[-0.200pt]{0.400pt}{4.818pt}}
\put(972,90){\makebox(0,0){1144}}
\put(972.0,570.0){\rule[-0.200pt]{0.400pt}{4.818pt}}
\put(1000.0,131.0){\rule[-0.200pt]{0.400pt}{4.818pt}}
\put(1000,90){\makebox(0,0){1225}}
\put(1000.0,570.0){\rule[-0.200pt]{0.400pt}{4.818pt}}
\put(191.0,131.0){\rule[-0.200pt]{0.400pt}{110.573pt}}
\put(191.0,131.0){\rule[-0.200pt]{194.888pt}{0.400pt}}
\put(1000.0,131.0){\rule[-0.200pt]{0.400pt}{110.573pt}}
\put(191.0,590.0){\rule[-0.200pt]{194.888pt}{0.400pt}}
\put(70,360){\makebox(0,0){Cover Size}}
\put(595,29){\makebox(0,0){Total Degree}}
\sbox{\plotpoint}{\rule[-0.600pt]{1.200pt}{1.200pt}}%
\put(191,370){\circle*{18}}
\put(219,296){\circle*{18}}
\put(247,223){\circle*{18}}
\put(275,296){\circle*{18}}
\put(303,260){\circle*{18}}
\put(330,333){\circle*{18}}
\put(358,443){\circle*{18}}
\put(386,370){\circle*{18}}
\put(414,443){\circle*{18}}
\put(442,480){\circle*{18}}
\put(470,443){\circle*{18}}
\put(498,517){\circle*{18}}
\put(526,590){\circle*{18}}
\put(554,553){\circle*{18}}
\put(582,553){\circle*{18}}
\put(609,517){\circle*{18}}
\put(637,553){\circle*{18}}
\put(665,553){\circle*{18}}
\put(693,517){\circle*{18}}
\put(721,553){\circle*{18}}
\put(749,553){\circle*{18}}
\put(777,590){\circle*{18}}
\put(805,590){\circle*{18}}
\put(833,590){\circle*{18}}
\put(861,590){\circle*{18}}
\put(888,553){\circle*{18}}
\put(916,553){\circle*{18}}
\put(944,553){\circle*{18}}
\put(972,590){\circle*{18}}
\put(1000,590){\circle*{18}}
\sbox{\plotpoint}{\rule[-0.400pt]{0.800pt}{0.800pt}}%
\put(191,168){\raisebox{-.8pt}{\makebox(0,0){$\Box$}}}
\put(219,278){\raisebox{-.8pt}{\makebox(0,0){$\Box$}}}
\put(247,149){\raisebox{-.8pt}{\makebox(0,0){$\Box$}}}
\put(275,204){\raisebox{-.8pt}{\makebox(0,0){$\Box$}}}
\put(303,223){\raisebox{-.8pt}{\makebox(0,0){$\Box$}}}
\put(330,241){\raisebox{-.8pt}{\makebox(0,0){$\Box$}}}
\put(358,370){\raisebox{-.8pt}{\makebox(0,0){$\Box$}}}
\put(386,333){\raisebox{-.8pt}{\makebox(0,0){$\Box$}}}
\put(414,443){\raisebox{-.8pt}{\makebox(0,0){$\Box$}}}
\put(442,406){\raisebox{-.8pt}{\makebox(0,0){$\Box$}}}
\put(470,425){\raisebox{-.8pt}{\makebox(0,0){$\Box$}}}
\put(498,517){\raisebox{-.8pt}{\makebox(0,0){$\Box$}}}
\put(526,498){\raisebox{-.8pt}{\makebox(0,0){$\Box$}}}
\put(554,553){\raisebox{-.8pt}{\makebox(0,0){$\Box$}}}
\put(582,517){\raisebox{-.8pt}{\makebox(0,0){$\Box$}}}
\put(609,517){\raisebox{-.8pt}{\makebox(0,0){$\Box$}}}
\put(637,553){\raisebox{-.8pt}{\makebox(0,0){$\Box$}}}
\put(665,553){\raisebox{-.8pt}{\makebox(0,0){$\Box$}}}
\put(693,553){\raisebox{-.8pt}{\makebox(0,0){$\Box$}}}
\put(721,535){\raisebox{-.8pt}{\makebox(0,0){$\Box$}}}
\put(749,553){\raisebox{-.8pt}{\makebox(0,0){$\Box$}}}
\put(777,535){\raisebox{-.8pt}{\makebox(0,0){$\Box$}}}
\put(805,553){\raisebox{-.8pt}{\makebox(0,0){$\Box$}}}
\put(833,553){\raisebox{-.8pt}{\makebox(0,0){$\Box$}}}
\put(861,572){\raisebox{-.8pt}{\makebox(0,0){$\Box$}}}
\put(888,553){\raisebox{-.8pt}{\makebox(0,0){$\Box$}}}
\put(916,572){\raisebox{-.8pt}{\makebox(0,0){$\Box$}}}
\put(944,572){\raisebox{-.8pt}{\makebox(0,0){$\Box$}}}
\put(972,572){\raisebox{-.8pt}{\makebox(0,0){$\Box$}}}
\put(1000,572){\raisebox{-.8pt}{\makebox(0,0){$\Box$}}}
\sbox{\plotpoint}{\rule[-0.200pt]{0.400pt}{0.400pt}}%
\put(191.0,131.0){\rule[-0.200pt]{0.400pt}{110.573pt}}
\put(191.0,131.0){\rule[-0.200pt]{194.888pt}{0.400pt}}
\put(1000.0,131.0){\rule[-0.200pt]{0.400pt}{110.573pt}}
\put(191.0,590.0){\rule[-0.200pt]{194.888pt}{0.400pt}}
\end{picture}

\end{figure}	

\bibliography{vertex_bib}
\end {document}
