\documentclass[conference, 11pt]{IEEEtran} 
\usepackage{verbatim}
\usepackage{multirow} \usepackage{enumerate}
\usepackage{amsmath,enumerate} \usepackage{amsthm}
\usepackage{algcompatible}
\usepackage{algpseudocode}
\usepackage{algorithm}
%\usepackage{algorithmic}
\usepackage{pstricks}
\usepackage{amssymb, latexsym}
\usepackage{xfrac}
\usepackage{mathtools}
\usepackage{graphicx}
\usepackage{subfig}
\DeclareGraphicsRule{*}{mps}{*}{}
\usepackage{listings}

%specific to this document only
\usepackage{pgfplots}
\usepackage{pgfplotstable}
\pgfplotstableread{plts/experiment8b1_av.tab}\averageone
\pgfplotstableread{plts/experiment8b2_av.tab}\averagetwo
\pgfplotstableread{plts/experiment8b3_av.tab}\averagethree
\pgfplotstableread{plts/experiment8b4_av.tab}\averagefour
\pgfplotstableread{plts/experiment9a_av.tab}\stepping
\pgfplotstableread{plts/experiment9a1_av.tab}\steppingone
\pgfplotstableread{plts/experiment9a2_av.tab}\steppingtwo
\pgfplotstableread{plts/experiment9a3_av.tab}\steppingthree
\pgfplotstableread{plts/experiment9a4_av.tab}\steppingfour
\pgfplotstableread{plts/experiment9b1_av.tab}\runningone
\pgfplotstableread{plts/experiment9b2_av.tab}\runningtwo
\pgfplotstableread{plts/experiment9b3_av.tab}\runningthree
\pgfplotstableread{plts/experiment9b4_av.tab}\runningfour
\pgfplotstableread{plts/experiment9b_av.tab}\running
\pgfplotstableread{plts/experiment9c_av.tab}\costcomp
\pgfplotstableread{plts/experiment9c1_av.tab}\costcompone
\pgfplotstableread{plts/experiment9c2_av.tab}\costcomptwo
\pgfplotstableread{plts/experiment9c3_av.tab}\costcompthree
\pgfplotstableread{plts/experiment8b1_rn.tab}\runsone
\pgfplotstableread{plts/experiment8b2_rn.tab}\runstwo
\pgfplotstableread{plts/experiment8b3_rn.tab}\runsthree
\pgfplotstableread{plts/experiment8b4_rn.tab}\runsfour
\pgfplotstableset{
  create on use/density/.style={
    create col/expr={\thisrow{nodes}+\thisrow{links}}}
    }
\pgfplotstableset{
  create on use/delta/.style={
    create col/expr={\thisrow{links}*2}}
    }
\pgfplotstableset{
  create on use/nodebylinks/.style={
    create col/expr={(\thisrow{nodes}*\thisrow{links})}}
    }
\pgfplotscreateplotcyclelist{three}{% 
  every mark/.append style={fill=teal}\\% 
  every mark/.append style={fill=green}\\% 
  every mark/.append style={fill=orange}\\% 
}
\pgfplotscreateplotcyclelist{four}{%
  every mark/.append style={fill=teal}\\%
  every mark/.append style={fill=green}\\%
  every mark/.append style={fill=orange}\\%
  every mark/.append style={fill=pink}\\%
}

%%%%%%%%%%%%%

\usepackage{pgf}
\usepackage{tikz}
\usetikzlibrary{decorations.pathmorphing} % LATEX and plain TEX when using Tik Z
\usetikzlibrary{positioning}
\usetikzlibrary{er}
\usetikzlibrary{automata}
\usetikzlibrary{shapes.geometric}
\tikzstyle{vx}=[draw,circle,fill=white,minimum size=2pt, inner sep=1pt, node distance=15mm]
\tikzstyle{ex}=[draw,rectangle,fill=white,minimum size=2pt, inner sep=3pt, node distance=15mm]
\tikzstyle{bup}=[semithick, decoration={bent, aspect=.3, amplitude=4}, decorate, ->, >=stealth]
\tikzstyle{bdn}=[semithick, decoration={bent, aspect=.3, amplitude=-4}, decorate, ->, >=stealth]
\tikzstyle{BUP}=[thick, decoration={bent, aspect=.3, amplitude=8}, decorate, ->, >=stealth]
\tikzstyle{BDN}=[thick, decoration={bent, aspect=.3, amplitude=-8}, decorate, ->, >=stealth]
\tikzstyle{MUP}=[thick, decoration={bent, aspect=.3, amplitude=16}, decorate, ->, >=stealth]
\tikzstyle{MDN}=[thick, decoration={bent, aspect=.3, amplitude=-16}, decorate, ->, >=stealth]
\tikzstyle{str}=[semithick, decorate, ->, >=stealth]
\tikzstyle{cr}=[draw, circle, fill=black!25,minimum size=150pt]

%styles for plots?
\tikzstyle{bls}=[blue, solid, mark=square*]
\tikzstyle{grt}=[red, solid, mark=*]
% \paperheight=11in \paperwidth=8.5in \textheight=9.0in
% \textwidth=6.5in \voffset=-.875in \hoffset=-.875in
\newenvironment{code} {\begin {quote}\begin{footnotesize}}
    {\end{footnotesize}\end{quote}}

% \oddsidemargin 0.0 in \evensidemargin 0.0 in
\newenvironment{enumeratealpha}
{\begin{enumerate}[(a{\textup{)}}]}{\end{enumerate}}

\theoremstyle{plain}
\newtheorem{lem-rule}{Rule}
\newtheorem{thm}{Theorem}
\newtheorem{lem}{Lemma}[thm]
\newtheorem{prop}{Proposition}[thm]
\newtheorem{lprp}{Proposition}[lem]
\theoremstyle{definition}
\newtheorem{defn}{Definition}[thm]
\newtheorem{dfn}{Definitions}[thm]
\newtheorem{ldef}{Definition}
\theoremstyle{remark}
\newtheorem{smy}{Summary}
\newtheorem{note}{Note}[thm]

%algorithms commands
\algblockdefx[Case]{Case}{EndCase} %
[1] [{\em var}] {{\bfseries case} {\em #1\ } } %
{{\bfseries end case}}%
\algcblockdefx[Case]{Case}{When}{EndCase}
[1] [{\em true}] {{\bfseries when} {\em #1\ }}
{{\bfseries end case}} %

\algblockdefx[TimesDo] {DoTimes}{EndTimes}
[1] [0] {#1 times {\bfseries do}}
{{\bfseries end do}}

%subalgorithms environment
\makeatletter
\newcounter{parentalgorithm}
\newenvironment{subalgorithms}{%
%  \refstepcounter{algorithm}%
  \floatname{algorithm}{Procedure}
  \protected@edef\theparentalgorithm{\thealgorithm}%
  \setcounter{parentalgorithm}{\value{algorithm}}%
  \setcounter{algorithm}{0}%
  \def\thealgorithm{\theparentalgorithm-\alph{algorithm}}%
  \ignorespaces
}{%
  \setcounter{algorithm}{\value{parentalgorithm}}%
  \ignorespacesafterend
}
\makeatother

%code environments
\usepackage{float}
 
\floatstyle{ruled}
\newfloat{codeblock}{thp}{lop}
\floatname{codeblock}{Example}

\lstnewenvironment{rubyblock} 
{\lstset{language=Ruby, basicstyle=\small, xleftmargin=10pt, numbers=left, numberstyle=\tiny, stepnumber=2, numbersep=5pt}}
{}
% text macros
\def\cI{{\mathcal I}} \def\cR{{\mathcal R}} \def\cE{{\mathcal E}}
\def\cC{{\mathcal C}} \def\cF{{\mathcal F}} \def\cU{{\mathcal U}}
\def\cH{{\mathcal H}} \def\cD{{\mathcal D}} \def\cB{{\mathcal B}}
\def\cQ{{\mathcal Q}} \def\cV{{\mathcal V}} \def\cS{{\mathcal S}}
\def\cG{{\mathcal G}} \def\cA{{\mathcal A}} \def\cO{{\mathcal O}}
\def\cW{{\mathcal W}} \def\cL{{\mathcal L}} 

\def\bI{{\mathbb I}} \def\bO{{\mathbb O}}
\def\bC{{\mathbb C}} \def\bM{{\mathbb M}}
\def\bId{{$\mathbb I$}} \def\bOd{{$\mathbb O$}}
\def\bCd{{$\mathbb C$}} \def\bMd{{$\mathbb M$}}

\def\cId{{$\mathcal I$}} \def\cRd{{$\mathcal R$}} \def\cEd{{$\mathcal E$}} 
\def\cCd{{$\mathcal C$}} \def\cFd{{$\mathcal F$}} \def\cUd{{$\mathcal U$}} 
\def\cHd{{$\mathcal H$}} \def\cDd{{$\mathcal D$}} \def\cBd{{$\mathcal B$}} 
\def\cQd{{$\mathcal Q$}} \def\cVd{{$\mathcal V$}} \def\cSd{{$\mathcal S$}} 
\def\cGd{{$\mathcal G$}} \def\cAd{{$\mathcal A$}} \def\cOd{{$\mathcal O$}}
\def\cWd{{$\mathcal W$}} \def\cLd{{$\mathcal L$}}

\bibliographystyle {IEEEtranS}

\begin {document}

\title{Generalized Maximal Matching for Extended Sensor Network Lifetimes} 

\author{\IEEEauthorblockN{J. Paul Daigle}
\IEEEauthorblockA{Department of Computer Science\\
Georgia State University\\
Atlanta, Georgia 30303\\
Email: jdaigle1@student.gsu.edu}
}

\maketitle

\begin{abstract}

\end{abstract}
\section{Generalized Maximal Matching}
The generalized maximal matching algorithm presented by Gonzalex is a linear time 2-approximation of the vertex cover problem that does not rely on linear programming or combinatorics.\cite{Gonzalez1995129} The algorithm proceeds by considering each edge of the graph in turn. For each edge, a weight is evaluated and one vertex of the edge will be denoted as {\em saturated} as described below. When each edge has been evaluated, the algorithm terminates and the saturated edges form a vertex cover.

\subsection{Definitions}
\begin{description}[\setlabelwidth{Incident Difference}]
\item [Graph] A vertex and edge weighted graph $G(V,E,W_v, W_e)$
\item [Neighborhood] $n_e$ for $v\in V$, such that $u \in n_v$ if $\exists e(u,v) \in E$
\item [Incident Weight] $i_v | i = w_v - \sum w_e \forall e(u,v) \forall u \in n_v$, that is, the sum of the weights of the edges incident to a vertex.
\item [Incident Difference] $d_v = w_v - i_v$
\end{description}

\subsection{Edge Weight Assignment}
\label{sec:sequential}
Using the above definitions, $w_e$ is set to zero for all $w_e\in W_e$. For each edge $e(u,v)$, leave the weight at 0 if either $u$ or $v$ are saturated. Otherwise, increase $w_e$ by the minimum incident difference between $u,v$ and mark the appropriate vertex as saturated. As one of these conditions must be met (either at least one node is saturated or not) for each edge, the edges only need be traversed once to complete the algorithm. As each edge is assigned a saturated vertex if it is without one when being considered, no edge is incident to two unsaturated nodes after being considered. Therefore, the saturated nodes will form a vertex cover on the Graph after each node has been considered once.

\subsection{Proof of 2-approximation}
Gonzalex presents a simple proof of 2-approximation based on the insight that the collection of saturated nodes can be divided into two sets (of which one might be empty), the set of all nodes that are part of the optimal cover and the set which are not. Both sets can be proven to be less or equal to the weight of the optimal cover.
\section{Distributed Generalized Maximal Matching}
\subsection{Assumptions}
The time complexity of our algorithm relies on several assumptions and limitations. We first assume that the network is initialized and that every node is aware of its neighbors. Second, we assume that some communication scheme has been established to allow nodes to communicate to their neighbors reliably, and that this communication scheme is bounded by the size of the local neighborhood rather than the total size of the network. We follow the literature in ignoring the questions of network initialization, collision, and collection and instead analyze only the specific communication and computation costs of the algorithm itself.

We rely on the established fact that a maximal matching is a vertex cover with an approximation of 2.\cite{1435381}  
\subsection{Distributed Algorithm}
\label{sub:alg-dgmm}
Algorithm~\ref{alg:dgmm} proceeds in steps. Each node in the network considers itself to have an edge with each of its neighbors. In each communication phase, one edge will be considered, that is, each node communicates reciprocally and exclusively with one of its neighbors to determine whether one of them should become part of the cover, based on the sequential algorithm described in section~\ref{sec:sequential}. We replace edge weight in the algorithm with {\em incident weight}, a value that begins as equal to the weight of the node and is decreased each time a node compares its current incident weight to any neighbors. When each node has communicated with each of its neighbors, the algorithm halts.

\begin{algorithm}
\caption{Distributed Weighted Vertex Cover}
\begin{algorithmic}
\Require {$G(V,E)$: a graph}
\ForAll {$v_u \in V$ in parrallel}
\State $S_u \leftarrow False$
\State state $\leftarrow$ Choose
\Repeat
\State Broadcast $S_u$
\If {$S_v = True$ for $v_v$ incident to self}
\State Set Weight $e_{u,v} \leftarrow 0$
\EndIf
\If {state = Choose}
\State {Choose A State (Invite, Listen)}
\ElsIf {state = Invite}
\State {Select an unweighted edge, $e_{u,v}$}
\State {Broadcast an Invitation to $v_v$}
\State {state $\leftarrow$ Wait}
\ElsIf {state = Listen}
\State {Collect Invitations}
\State {state $\leftarrow$ Respond}
\ElsIf {state = Wait}
\State {Collect Responses}
\If {Response Matches Invitation}
\State {Update Weight $e_{u,v}$}
\If {$\sum_{w_e} e incident v_u = w_u$}
\State $S_u \leftarrow true$
\EndIf
\EndIf
\State {state $\leftarrow$ Choose}
\ElsIf {state = Respond}
\State Choose Invitation, Broadcast Response
\State {Update Weight $e_{u,v}$}
\If {$\sum_{w_e} e incident v_u = w_u$}
\State $S_u \leftarrow true$
\EndIf
\State {state $\leftarrow$ Choose}
\EndIf
\Until {$S_u = true$ OR $S_v = true$ for all $v_v$ incident $v_u$}
\EndFor
\end{algorithmic}
\label{alg:dgmm}
\end{algorithm}

The distributed algorithm can be viewed as a special ordering of the sequential algorithm. Each pair of vertexes examines only one edge in a communication round. These edges are disjoint, that is, no two edges being examined share a common vertex, so the weighting of any edge in a given round is independent of the weighting of any other edge evaluated in that round. Establishing a weight for each of these edges and determining whether any endpoint fits into the cover in parallel is therefore equivalent to making the same determination for each edge in sequence. From this it follows that if the sequential algorithm is correct and produces a 2-approximate minimum weighted vertex cover, the distributed algorithm is also correct and also produces a 2-approximate minimum weighted vertex cover.   

\section{Extension to Hypergraphs}
\subsection{Equivalence to Target Coverage}
\begin{thm}
  For a given network $N(S,T)$ with sensors $s\in S$ and targets $t\in T$, there exists an equivalent hypergraph $G(V,E)$ such that any solution to the target coverage problem in $N$ is a vertex cover in $G$, and any vertex cover of $G$ is a target cover for $N$.
\label{thm:equiv}
\end{thm}
\begin{proof}[Proof of Theorem~\ref{thm:equiv}]
\begin{prop}
For $N(S,T)$, define set of all sensors which can cover a target as $c(t_i)$ = \{$s \in S | $s$ covers $t$\}.  For example, if sensors $s_1, s_4, s_7$ are all of the sensors that cover $t_3$, $c(t_3) = \{s_1, s_4, s_5\}$. \label{prp:network}
\end{prop}
\begin{prop}To construct the equivalent hypergraph $G(V,E)$, as $V = S$ and $E$ is the set of hyper edges $c(t_i)$, for all $t_i \in T$. 
\label{prp:graph}
\end{prop}

To be shown: if $S' \subset S$ is a solution for target coverage in $N$, $f(S')$ is an edge cover of $G$. 

We proceed by direct proof. Let us define $S'\subset S$ such that $S'$ is a target cover of $N$. If $S'$ is a target cover, we know that 
\begin{equation}
\forall t \in T, \exists s \in t | s \in S'
\label{eqn:tcover}
\end{equation}
That is, for every target, there is a sensor in $S'$ that covers that target.

We also know that there exists an equivalent edge in $G$ for each target, defined as $g(t)$. We rewrite equation~\ref{eqn:tcover} to include proposition~\ref{prp:graph},
\begin{equation}
\forall t \in T, \exists s \in t | s \in S' \land f(s) \in g(t)
\label{eqn:extcover}
\end{equation}

Since the range of $g$ is comprehensive of $E$, this implies that for all edges,
\begin{equation}
\forall e \in E, \exists s \in S' | f(s) \in e
\end{equation}
Therefore, if we apply $f(s)$ to every $s \in S'$, we will create a vertex set $V'$ such that
\begin{equation}
\forall e \in E, \exists v \in V' | v \in e
\label{eqn:vcover}
\end{equation}
Equation~\ref{eqn:vcover} describes a vertex cover of $G$, which is what we wanted to find.

The second part of the theorem follows from the definition of bijection. 
\end{proof}

\subsection{The Algorithm for Edge Equivalence in Hypergraphs}
To design the hypergraph algorithm, we assume that each vertex is aware of the membership of each of its edges. Each vertex will therefore attempt to establish an exclusive communication clique for each of its hyperedge. One possible method of doing so would be to hash the edges according to unique sensor ids, assuring that each edge in the graph has a unique, sortable identifier without requiring network wide communication.

We define the weight of each vertex $v \in V$ as $\omega_v$ and the {\em incident weight} of $v$ as $\varpi_v$. If two vertexes have equal incident weight, then ties are broken by choosing the vertex with the lowest index.

\begin{algorithm}
\caption{Distributed Hypergraph General Maximal Matching} 
\begin{algorithmic}
\REQUIRE $\forall v \in V \varpi_v \leftarrow \omega_v$
\REQUIRE Saturation State $s_v \leftarrow \text{FALSE}$ 
\FORALL {$u \in V$ in parrallel}
\FORALL {$e\{u, v_0, v_1, v_2\}$}
\STATE Form an exclusive communication clique $u, v_0, v_1, ... v_n$
\IF {$\forall v\in e$, $s_v = \text{FALSE}$}
\STATE Find the lowest incident weight $min_\varpi$
\FORALL {$v \in e$}
\STATE $\varpi_v \leftarrow \varpi_v - \min_\varpi$
\IF {$varpi_v = 0$} 
\STATE $s_v \leftarrow \text{TRUE}$ 
\ENDIF
\ENDFOR
\ENDIF
\ENDFOR
\ENDFOR
\end{algorithmic}
\label{alg:dhgmm}
\end{algorithm}

When the algorithm has finished, the edges that are saturated form a cover.
\subsection{Correctness of Algorithm~\ref{alg:dhgmm}}
Algorithm~\ref{alg:dhgmm} is correct if at least one vertex for each hyperedge is saturated. It is trivial to see that the algorithm is correct, each edge in the graph is selected, and the incident weight of each vertex in the edge is decreased by the amount necessary to saturate at least one edge.
\subsection{2-Approximation of the Algorithm}
A slight modification of the proof in \cite{Gonzalez1995129} produces a proof that Algorithm~\ref{alg:dhgmm} is a 2-approximation for edge cover in hypergraphs.


\begin{thm}
  Algorithm~\ref{alg:dhgmm} produces a 2-approximate edge cover.
  \label{thm:dhgmm}
\end{thm}
\begin{proof}{Proof of Theorem~\ref{thm:dhgmm}}
  \begin{smy}
    The outline of the proof is as follows. Algorithm~\ref{alg:dhgmm} produces a cover of the hypergraph $G$, which we will define as \cAd. We would like to show that \cAd\ is a 2-approximation of \cCd\, the optimal cover of $G$. In order to show that this is the case, we divide the appoximate cover into two non-overlapping sets. The first set, \bId, conists of the vertexes shared by \cAd\ and \cCd\. Because \bId\ is a subset of \cCd\, we know that the weight of \bId\ is less than or equal to the weight of \cCd. 

The second set, \bOd\, consists of vertexes that are in the approximate cover, but not in the optimal cover. We would like to show that the weight of \bOd\ is less than or equal to the weight of \cCd. If it is, than the weight of the approximate cover is no greater than twice the weight of the optimal cover, and we have proved that \cAd\ is a two-approximation of \cCd\, which is what we want to show. This can be proven by using the {\em incident weight} of the vertexes, as defined in section~\ref{sub:alg-dgmm}. 

\cAd\ can be divided into two sets, \bId\ and \bOd, that is, those vertexes that are in \cAd\ and also in the optimal edge cover, and those vertexes that are in \cAd\ but are not in the optimal edge cover. In the formal proof below, the optimal cover is referred to by \cCd. \footnote{The mnemonics for this proof are as follows: we are using calligraphic letters to refer to the {\em Approximate} cover and the optimal {\em Cover}. The approximate cover is divided into the two sets denoted by blackboard style, those vertexes that are {\em In} the optimal cover and those that are {\em Out} of the optimal cover. Weight of any set or vertex is denoted with an $\omega$ because of the similarity to w on the one hand, but also because the symbol $\varpi$ looks like a w plus an edge, which conceptually relates to the concept of {\em incident weight}. Incidence is illustrated by placing a line over the incident vertexes.} What we would like to show is that both the set of vertexes that are in both \cAd\ and \cCd\ and the set of vertexes that are exclusive to \cAd\ have a total weight less than or equal to that of the optimal cover. If this is true, than the sum of the weights of the two sets is less than or equal to twice the weight of the optimal cover, and Algorithm~\ref{alg:dhgmm} produces a two approximate cover.
  \end{smy}
    
  \begin{dfn}
    \begin{description}
    \item[$G$] A hypergraph $G(V,E)$ with vertexes $v \in V$ and edges $e\{u,v,...,z\} \in E$
    \item[\cCd] The optimal edge cover of $G$
    \item[\cAd] The cover generated by Algorithm~\ref{alg:dhgmm}
    \item[\bId] $v \in \cA \land v\in \cC$
    \item[\bOd] $v \in \cA \land v\notin \cC$
    \item[$\omega_u$] The weight of u
    \item[$\varpi_v$] The incident weight of v
    \item[$ \overline{uv} $] u is incident to v in $G$
    \end{description}
  \end{dfn}
  \begin{prop}
    $\omega_{\bI} \le \omega_{\cC} \land \omega_{\bO} \le \omega_{\cC} \implies \omega_{\cA} \le 2(\omega_{\cC})$, which is what must be shown as explained in the summary.
    \label{prp:weightcomp}
  \end{prop}

  \begin{lem}
    $\omega_{\bI} \le \omega_{\cC}$
    \label{lem:IleC}
  \end{lem}
  \begin{proof}{Proof of Lemma~\ref{lem:IleC}}
    This is obvious, as no vertex can be counted twice, and every vertex that is in \bId\ is also in \cCd. So the sum of the weights of the vertexes that are in both the generated cover and the optimal cover is less than the sum of the weights of the optimal cover.
  \end{proof}
  \begin{lem}
    $\omega_{\bO} \le \omega_{\cC}$
    \label{lem:OleC}
  \end{lem}
  \begin{note}
    Lemma~\ref{lem:OleC} states that the sum of the weights of the vertexes that are in the Aproximate cover but not in the optimal cover is less than or equal to the sum of the weights of the optimal cover.
  \end{note}
  \begin{lem}
    $\forall u \in C$, $\sum_{\omega_v} (v \in \bO\text{, }\overline{vu}) \le \omega_u$
    \label{lem:incidentweight}
  \end{lem}
  \begin{note}
    Lemma~\ref{lem:incidentweight} specifies Lemma~\ref{lem:OleC} to a single vertex in the optimal cover. 
  \end{note} 
  \begin{proof}{Proof of Lemma~\ref{lem:OleC},~\ref{lem:incidentweight}}
    \begin{equation}
      v \in \bO \land v \notin \cC \implies \overline{uv} \text{, } u \in \cC
    \end{equation}
    Assume Lemma~\ref{lem:incidentweight} is false. In this case, $\varpi_u \le 0$. However, in order for this to happen, a vertex $v\sigma u$ would have been chosen by Algorithm~\ref{alg:dhgmm} when it was both being compared to $u$ and $\varpi_v > \varpi_u$. This contradicts the basic function of the algorithm, Lemma~\ref{lem:incidentweight} is true.
    Lemma~\ref{lem:incidentweight} implies $\omega_{\bO} \le \omega_{\cC}$, so Lemma~\ref{lem:OleC} is true as well.
  \end{proof}
        
  Therefore, the conditions of proposition~\ref{prp:weightcomp} are satisfied, and Thereom~\ref{thm:dhgmm} is true.
\end{proof}


\section{Analysis of Algorithm~\ref{alg:dgmm} and Algorithm~\ref{alg:dhgmm}}
There are two factors to be considered in the analysis of the algorithms. The first is the difficulty of finding partners in an ad hoc network. This problem is related to channel allocation, that is, each network node wants to communicate to each of it's neighbors, and each such communication must be scheduled in some conflict free manner. This problem has been well studied, and the communication delay for the network can be calculated to be no more than $O(\Delta^{\Delta})$ for the pair matching and communication phase to complete for all nodes in the graph.\cite{4053985}

For the purpose of analysis, Algorithm~\ref{alg:dgmm} can be treated as a special case of Algorithm~\ref{alg:dhgmm} in which each hyperedge contains only two vertexes.
 
It is trivial to show that the communication time of the algorithm is bounded by the degree of the graph, as each vertex communicates only with its neighbors and communicates with each neighbor only once. This communication is a constant, that is, each vertex simply communicates its incident weight to its partner for the round. The vertices can then independently determine the correct weight to assign to the edge, determine their saturation states, and update their incident weights without any further information being exchanged.

Each vertex $v$ must visit each of its edges exactly once. The total number of edges that any vertex has to visit is therefore bounded by its degree. In each communication round, there are two possibilities. Either $v$ finds all of its partners for some edge or it does not. In the first case, $v$ will reduce the number of remaining edges to be visited by exactly one. In the second case, at least one vertex $u$ in each hyperedge of $v$ must have formed a communication clique, reducing the number of edges that $u$ must visit by exactly 1. Assume that $v$ does not form a partnership until all of its neighbors have updated all of their other edges. The algorithm will have run for $\Delta_p$ rounds, where $\Delta_p$ is the largest degree of a vertex in the neighborhood of $v$. $v$ must now form a communication clique for each of its edges, and the algorithm will continue for $\Delta_v$ rounds, where $\Delta_v$ is the degree of $v$. Assuming that both $\Delta_l$ and $\Delta_p$ are the highest degrees in the Graph, we see that the algorithm is still bounded by $O(\Delta)$ in the worst case.

Algorithm~\ref{alg:dhgmm} is the first example in the literature of a fully distributed 2-approximation for the minimum vertex cover problem in Graphs and Hypergraphs that is bounded by $\Delta$ rather than the size of the graph.\cite{1435381}

\section{Results} 
In this section, we evaluate the performance of our algorithm in two cases. First, we compare the performance of Distributed GMM to sequential GMM. Second, we confirm the $O(\Delta)$ communication rounds experimentally. 
\subsection{Design}
Both algorithms were coded in Python.\footnote{Software is licensed under an MIT open license and available for download from Google Code} Graphs were generated with random topology. For each experimental run, two separate cases are evaluated. In the first the nodes of the graph are kept constant with the average degree of the graph increasing until a clique is generated. In the second, the average degree of the graph is kept constant while the number of nodes is increased.
\subsection{Results}
\begin{center}
\begin{figure}[width=2in]
  \label{fig:unweight_covers}
  \caption{Unweighted Case Versus 2-approximation}
  % GNUPLOT: LaTeX picture
\setlength{\unitlength}{0.240900pt}
\ifx\plotpoint\undefined\newsavebox{\plotpoint}\fi
\begin{picture}(1050,630)(0,0)
\sbox{\plotpoint}{\rule[-0.200pt]{0.400pt}{0.400pt}}%
\put(191.0,131.0){\rule[-0.200pt]{4.818pt}{0.400pt}}
\put(171,131){\makebox(0,0)[r]{ 25}}
\put(980.0,131.0){\rule[-0.200pt]{4.818pt}{0.400pt}}
\put(191.0,223.0){\rule[-0.200pt]{4.818pt}{0.400pt}}
\put(171,223){\makebox(0,0)[r]{ 30}}
\put(980.0,223.0){\rule[-0.200pt]{4.818pt}{0.400pt}}
\put(191.0,315.0){\rule[-0.200pt]{4.818pt}{0.400pt}}
\put(171,315){\makebox(0,0)[r]{ 35}}
\put(980.0,315.0){\rule[-0.200pt]{4.818pt}{0.400pt}}
\put(191.0,406.0){\rule[-0.200pt]{4.818pt}{0.400pt}}
\put(171,406){\makebox(0,0)[r]{ 40}}
\put(980.0,406.0){\rule[-0.200pt]{4.818pt}{0.400pt}}
\put(191.0,498.0){\rule[-0.200pt]{4.818pt}{0.400pt}}
\put(171,498){\makebox(0,0)[r]{ 45}}
\put(980.0,498.0){\rule[-0.200pt]{4.818pt}{0.400pt}}
\put(191.0,590.0){\rule[-0.200pt]{4.818pt}{0.400pt}}
\put(171,590){\makebox(0,0)[r]{ 50}}
\put(980.0,590.0){\rule[-0.200pt]{4.818pt}{0.400pt}}
\put(191.0,131.0){\rule[-0.200pt]{0.400pt}{4.818pt}}
\put(191,90){\makebox(0,0){50}}
\put(191.0,570.0){\rule[-0.200pt]{0.400pt}{4.818pt}}
\put(219.0,131.0){\rule[-0.200pt]{0.400pt}{4.818pt}}
\put(219,90){\makebox(0,0){51}}
\put(219.0,570.0){\rule[-0.200pt]{0.400pt}{4.818pt}}
\put(247.0,131.0){\rule[-0.200pt]{0.400pt}{4.818pt}}
\put(247,90){\makebox(0,0){52}}
\put(247.0,570.0){\rule[-0.200pt]{0.400pt}{4.818pt}}
\put(275.0,131.0){\rule[-0.200pt]{0.400pt}{4.818pt}}
\put(275,90){\makebox(0,0){53}}
\put(275.0,570.0){\rule[-0.200pt]{0.400pt}{4.818pt}}
\put(303.0,131.0){\rule[-0.200pt]{0.400pt}{4.818pt}}
\put(303,90){\makebox(0,0){54}}
\put(303.0,570.0){\rule[-0.200pt]{0.400pt}{4.818pt}}
\put(330.0,131.0){\rule[-0.200pt]{0.400pt}{4.818pt}}
\put(330,90){\makebox(0,0){55}}
\put(330.0,570.0){\rule[-0.200pt]{0.400pt}{4.818pt}}
\put(358.0,131.0){\rule[-0.200pt]{0.400pt}{4.818pt}}
\put(358,90){\makebox(0,0){97}}
\put(358.0,570.0){\rule[-0.200pt]{0.400pt}{4.818pt}}
\put(386.0,131.0){\rule[-0.200pt]{0.400pt}{4.818pt}}
\put(386,90){\makebox(0,0){98}}
\put(386.0,570.0){\rule[-0.200pt]{0.400pt}{4.818pt}}
\put(414.0,131.0){\rule[-0.200pt]{0.400pt}{4.818pt}}
\put(414,90){\makebox(0,0){140}}
\put(414.0,570.0){\rule[-0.200pt]{0.400pt}{4.818pt}}
\put(442.0,131.0){\rule[-0.200pt]{0.400pt}{4.818pt}}
\put(442,90){\makebox(0,0){141}}
\put(442.0,570.0){\rule[-0.200pt]{0.400pt}{4.818pt}}
\put(470.0,131.0){\rule[-0.200pt]{0.400pt}{4.818pt}}
\put(470,90){\makebox(0,0){183}}
\put(470.0,570.0){\rule[-0.200pt]{0.400pt}{4.818pt}}
\put(498.0,131.0){\rule[-0.200pt]{0.400pt}{4.818pt}}
\put(498,90){\makebox(0,0){225}}
\put(498.0,570.0){\rule[-0.200pt]{0.400pt}{4.818pt}}
\put(526.0,131.0){\rule[-0.200pt]{0.400pt}{4.818pt}}
\put(526,90){\makebox(0,0){226}}
\put(526.0,570.0){\rule[-0.200pt]{0.400pt}{4.818pt}}
\put(554.0,131.0){\rule[-0.200pt]{0.400pt}{4.818pt}}
\put(554,90){\makebox(0,0){268}}
\put(554.0,570.0){\rule[-0.200pt]{0.400pt}{4.818pt}}
\put(582.0,131.0){\rule[-0.200pt]{0.400pt}{4.818pt}}
\put(582,90){\makebox(0,0){310}}
\put(582.0,570.0){\rule[-0.200pt]{0.400pt}{4.818pt}}
\put(609.0,131.0){\rule[-0.200pt]{0.400pt}{4.818pt}}
\put(609,90){\makebox(0,0){352}}
\put(609.0,570.0){\rule[-0.200pt]{0.400pt}{4.818pt}}
\put(637.0,131.0){\rule[-0.200pt]{0.400pt}{4.818pt}}
\put(637,90){\makebox(0,0){394}}
\put(637.0,570.0){\rule[-0.200pt]{0.400pt}{4.818pt}}
\put(665.0,131.0){\rule[-0.200pt]{0.400pt}{4.818pt}}
\put(665,90){\makebox(0,0){436}}
\put(665.0,570.0){\rule[-0.200pt]{0.400pt}{4.818pt}}
\put(693.0,131.0){\rule[-0.200pt]{0.400pt}{4.818pt}}
\put(693,90){\makebox(0,0){478}}
\put(693.0,570.0){\rule[-0.200pt]{0.400pt}{4.818pt}}
\put(721.0,131.0){\rule[-0.200pt]{0.400pt}{4.818pt}}
\put(721,90){\makebox(0,0){561}}
\put(721.0,570.0){\rule[-0.200pt]{0.400pt}{4.818pt}}
\put(749.0,131.0){\rule[-0.200pt]{0.400pt}{4.818pt}}
\put(749,90){\makebox(0,0){603}}
\put(749.0,570.0){\rule[-0.200pt]{0.400pt}{4.818pt}}
\put(777.0,131.0){\rule[-0.200pt]{0.400pt}{4.818pt}}
\put(777,90){\makebox(0,0){645}}
\put(777.0,570.0){\rule[-0.200pt]{0.400pt}{4.818pt}}
\put(805.0,131.0){\rule[-0.200pt]{0.400pt}{4.818pt}}
\put(805,90){\makebox(0,0){728}}
\put(805.0,570.0){\rule[-0.200pt]{0.400pt}{4.818pt}}
\put(833.0,131.0){\rule[-0.200pt]{0.400pt}{4.818pt}}
\put(833,90){\makebox(0,0){770}}
\put(833.0,570.0){\rule[-0.200pt]{0.400pt}{4.818pt}}
\put(861.0,131.0){\rule[-0.200pt]{0.400pt}{4.818pt}}
\put(861,90){\makebox(0,0){853}}
\put(861.0,570.0){\rule[-0.200pt]{0.400pt}{4.818pt}}
\put(888.0,131.0){\rule[-0.200pt]{0.400pt}{4.818pt}}
\put(888,90){\makebox(0,0){895}}
\put(888.0,570.0){\rule[-0.200pt]{0.400pt}{4.818pt}}
\put(916.0,131.0){\rule[-0.200pt]{0.400pt}{4.818pt}}
\put(916,90){\makebox(0,0){978}}
\put(916.0,570.0){\rule[-0.200pt]{0.400pt}{4.818pt}}
\put(944.0,131.0){\rule[-0.200pt]{0.400pt}{4.818pt}}
\put(944,90){\makebox(0,0){1061}}
\put(944.0,570.0){\rule[-0.200pt]{0.400pt}{4.818pt}}
\put(972.0,131.0){\rule[-0.200pt]{0.400pt}{4.818pt}}
\put(972,90){\makebox(0,0){1144}}
\put(972.0,570.0){\rule[-0.200pt]{0.400pt}{4.818pt}}
\put(1000.0,131.0){\rule[-0.200pt]{0.400pt}{4.818pt}}
\put(1000,90){\makebox(0,0){1225}}
\put(1000.0,570.0){\rule[-0.200pt]{0.400pt}{4.818pt}}
\put(191.0,131.0){\rule[-0.200pt]{0.400pt}{110.573pt}}
\put(191.0,131.0){\rule[-0.200pt]{194.888pt}{0.400pt}}
\put(1000.0,131.0){\rule[-0.200pt]{0.400pt}{110.573pt}}
\put(191.0,590.0){\rule[-0.200pt]{194.888pt}{0.400pt}}
\put(70,360){\makebox(0,0){Cover Size}}
\put(595,29){\makebox(0,0){Total Degree}}
\sbox{\plotpoint}{\rule[-0.600pt]{1.200pt}{1.200pt}}%
\put(191,370){\circle*{18}}
\put(219,296){\circle*{18}}
\put(247,223){\circle*{18}}
\put(275,296){\circle*{18}}
\put(303,260){\circle*{18}}
\put(330,333){\circle*{18}}
\put(358,443){\circle*{18}}
\put(386,370){\circle*{18}}
\put(414,443){\circle*{18}}
\put(442,480){\circle*{18}}
\put(470,443){\circle*{18}}
\put(498,517){\circle*{18}}
\put(526,590){\circle*{18}}
\put(554,553){\circle*{18}}
\put(582,553){\circle*{18}}
\put(609,517){\circle*{18}}
\put(637,553){\circle*{18}}
\put(665,553){\circle*{18}}
\put(693,517){\circle*{18}}
\put(721,553){\circle*{18}}
\put(749,553){\circle*{18}}
\put(777,590){\circle*{18}}
\put(805,590){\circle*{18}}
\put(833,590){\circle*{18}}
\put(861,590){\circle*{18}}
\put(888,553){\circle*{18}}
\put(916,553){\circle*{18}}
\put(944,553){\circle*{18}}
\put(972,590){\circle*{18}}
\put(1000,590){\circle*{18}}
\sbox{\plotpoint}{\rule[-0.400pt]{0.800pt}{0.800pt}}%
\put(191,168){\raisebox{-.8pt}{\makebox(0,0){$\Box$}}}
\put(219,278){\raisebox{-.8pt}{\makebox(0,0){$\Box$}}}
\put(247,149){\raisebox{-.8pt}{\makebox(0,0){$\Box$}}}
\put(275,204){\raisebox{-.8pt}{\makebox(0,0){$\Box$}}}
\put(303,223){\raisebox{-.8pt}{\makebox(0,0){$\Box$}}}
\put(330,241){\raisebox{-.8pt}{\makebox(0,0){$\Box$}}}
\put(358,370){\raisebox{-.8pt}{\makebox(0,0){$\Box$}}}
\put(386,333){\raisebox{-.8pt}{\makebox(0,0){$\Box$}}}
\put(414,443){\raisebox{-.8pt}{\makebox(0,0){$\Box$}}}
\put(442,406){\raisebox{-.8pt}{\makebox(0,0){$\Box$}}}
\put(470,425){\raisebox{-.8pt}{\makebox(0,0){$\Box$}}}
\put(498,517){\raisebox{-.8pt}{\makebox(0,0){$\Box$}}}
\put(526,498){\raisebox{-.8pt}{\makebox(0,0){$\Box$}}}
\put(554,553){\raisebox{-.8pt}{\makebox(0,0){$\Box$}}}
\put(582,517){\raisebox{-.8pt}{\makebox(0,0){$\Box$}}}
\put(609,517){\raisebox{-.8pt}{\makebox(0,0){$\Box$}}}
\put(637,553){\raisebox{-.8pt}{\makebox(0,0){$\Box$}}}
\put(665,553){\raisebox{-.8pt}{\makebox(0,0){$\Box$}}}
\put(693,553){\raisebox{-.8pt}{\makebox(0,0){$\Box$}}}
\put(721,535){\raisebox{-.8pt}{\makebox(0,0){$\Box$}}}
\put(749,553){\raisebox{-.8pt}{\makebox(0,0){$\Box$}}}
\put(777,535){\raisebox{-.8pt}{\makebox(0,0){$\Box$}}}
\put(805,553){\raisebox{-.8pt}{\makebox(0,0){$\Box$}}}
\put(833,553){\raisebox{-.8pt}{\makebox(0,0){$\Box$}}}
\put(861,572){\raisebox{-.8pt}{\makebox(0,0){$\Box$}}}
\put(888,553){\raisebox{-.8pt}{\makebox(0,0){$\Box$}}}
\put(916,572){\raisebox{-.8pt}{\makebox(0,0){$\Box$}}}
\put(944,572){\raisebox{-.8pt}{\makebox(0,0){$\Box$}}}
\put(972,572){\raisebox{-.8pt}{\makebox(0,0){$\Box$}}}
\put(1000,572){\raisebox{-.8pt}{\makebox(0,0){$\Box$}}}
\sbox{\plotpoint}{\rule[-0.200pt]{0.400pt}{0.400pt}}%
\put(191.0,131.0){\rule[-0.200pt]{0.400pt}{110.573pt}}
\put(191.0,131.0){\rule[-0.200pt]{194.888pt}{0.400pt}}
\put(1000.0,131.0){\rule[-0.200pt]{0.400pt}{110.573pt}}
\put(191.0,590.0){\rule[-0.200pt]{194.888pt}{0.400pt}}
\end{picture}

\end{figure}	
\end{center}
\bibliography{vertex_bib}
\end {document}
