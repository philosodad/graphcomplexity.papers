\section{Conclusion}
\label{sec:conclusion}

In this work we have presented a new distributed algorithm for the Minimum Weighted Vertex Cover problem, a well known NP-Complete problem with known approximation bounds in the distributed case\cite{1011811}. Our Distributed Generalized Maximal Matching algorithm, based on the sequential algorithm presented in \cite{Gonzalez1995129}, runs in $O(\log n)$ communication bounds for random graphs, which improves on the previous best known running time of $O(\log n)$ communication rounds\cite{1582746}. We tested the performance of our algorithm through simulation and confirmed this prediction.

This work led us to explore an optimization routine for removing redundant nodes. This routine had only modest performance for MWVC. However, we were able to successfully use this routine as a basis for making local adjustments for the Network Lifetime problem in Sensor Networks. Our approach drastically reduces the number of global adjustments that a sensor network would need to make in order to maintain coverage of a set of targets. Through simulation, we tested our approach against the global approach, using both the DEEPS algorithm\cite{1640702} and a novel algorithm of our own design. In both cases, we found that the local adjustments would lead to longer network lifetimes if the cost of global adjustment was greater than forty percent of the total network maintenance cost.

We anticipate two threads of future work, both theoretical and practical. In theoretical work, first, our DGMM algorithm will take $0(\Delta)$ rounds to terminate in the worst case. We have spent some time determining what inputs will generate this behavior, and designed modifications that should covert these cases to constant time, improving the performance of the algorithm in these cases. One such case is a weighted small-world graph with weights biased towards high degree nodes. Work by \cite{PhysRevE.65.061910} suggests that this may be of interest in the biological sciences. Second, we intend to extend DGMM to the hypergraph, both for general interest and as a solution to the target coverage problem. For practical application, the question of how much the global reshuffling of sensor nodes costs as a percentage of total network communication costs is open. We believe that experimental work using our local reshuffling strategy might yield interesting results.
