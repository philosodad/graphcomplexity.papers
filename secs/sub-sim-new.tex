
\label{sec:sim-new}

In this section, we create a new simulator to solve the Maximum Weighted Independent Set problem. It should be noted that as this is an example, we are not concerned about the quality of our algorithm as long as it produces an independent set. We will apply the dependency graph framework in our algorithm.

\paragraph{Problem Definition}
Given an undirected graph G(V,E), an {\bfseries Independent Set} of G is a subset $V' \subset V | \forall e(u,v) \in E, u \in V' \implies v \notin V'$. A maximum independent set would be an Independent set of G with the maximum possible nodes, and the maximum weighted independent set is an inpendent set of a vertex weighted graph such that $\sum_{v \in V'} w(v)$ is the maximum possible over all independent sets of G.

\paragraph{Approach}

We start by formulating an algorithm to solve the problem. In this case, we will use the Dependency Graph framework to solve the problem, passing over the question of whether this problem can be solved locally at all. Because much of our design will be based on a finite state machine model, we begin by defining the needed communication and action steps to move each independent node towards a finished state. Figure~\ref{fig:isdg-alg} shows the steps for Algorithm~\ref{alg:alg-isdg}.

\begin{enumerate}
\item Each item {\em gathers} (state \cGd) the information for its two-hop neighborhood.
\item Each node {\em builds} (state \cBd) all possible independent sets out to its two hop neighbors.
\item Each node ranks its independent sets, first on weight and then by shared nodes with other sets.
\item Use the sets to {\em analyze} whether to turn on and off according to these rules: \label{item:analyze}
\begin{enumerate}
\item If a node is in the heaviest local node in the best possible set, it turns on.
\item If a neighbor turns on, turn off.
\end{enumerate}
\item If neither on nor off, update the dependency graph (state \cUd) 
\item Send neighborhood information and repeat from step~\ref{item:analyze}
\item When all nodes are decided, complete.
\end{enumerate}

\begin{figure}[htp]
  \begin{center}
    \begin{tikzpicture}
     % \draw[help lines] (0, -2) grid (5,2);
      \path [help lines] (0, -2) grid (5,2);
      \node [state, initial] (G) {\cGd};
      \node [state] (B) at (1.5, 0) {\cBd};
      \node [state] (A) at (3, 0) {\cAd};
			\node [state] (W) at (4.5, 0) {\cWd};
			\node [state] (U) at (4.5, 1.5) {\cUd};
      \node [state, accepting] (I) at (3, 1.5) {\cId};
      \node [state, accepting] (O) at (3, -1.5) {\cOd};

      \path [->] (G) edge node [above] {2} (B)
										 edge [loop above] node {1} ()
                 (B) edge node [above] {b} (A)
                 (A) edge node [left]{o} (O)
                     edge node [left]{j} (I)
										 edge node [below] {w} (W)
								 (W) edge [bend right] node [right] {s} (U)
                 (U) edge node [right] {s} (A)
								 (O) edge [loop right] node {s} ()
								 (I) edge [loop left] node {s} ();
      \node [text width=8cm] (key) at (2, -2.25) {
        $1$: Gathered neighbors once  \hspace{7cm}
        $2$: Gathered neighbors twice  \hspace{7cm}
        $b$: Sets built\hspace{7cm}
        $j$: Best in best cover \hspace{9cm}
        $w$: Wait for more information\hspace{7cm}
        $o$: One neighbor on\hspace{7cm}
        $s$: Share onlist and offlist\hspace{7cm}};

    \end{tikzpicture}
    \caption{Independent Set Dependency Algorithm}

    \label{fig:isdg-alg}
  \end{center}
\end{figure}


In the end, we should end up with an independent set.

\paragraph{Classes}

To build this simulator we will need a minimum of~\ref{en:classlist-lastline} classes.
\begin{enumerate}
\item A Dependency Graph Node.
\item A Dependency Graph.
\item A Node.
\item A Graph.
\item A Simulator.\label{en:classlist-lastline}
\end{enumerate}

In this case, we will start in the middle and design the Node first. Looking at Figure~\ref{fig:node-class}, the {\bfseries BasicNode} has most of what we will need. We only need to add an attribute for building the Dependency Graph, and probably a method to do so. Looking through the types that we have, we realize that {\bfseries PCDRoot} (Figure~\ref{fig:pcd-node-class}) is actually an appropriate starting point for our new node type. It has an appropriate constructor and appropriate attributes, even though the variable {\em covers} and the method {\em init\_covers} have names that are less generic than we might like. We can write a new Node type that inherits this in just a few lines by overloading {\ttfamily get\_dep\_graph\_type}. Example~\ref{exa:isdg-1} shows the new class definition.

\begin{codeblock}
\caption{Class Definition of ISDG Node}
\begin{rubyblock}
class ISDGRoot < PCDROOT
  def init_covers
    return IS_Graph.new()
  end
end
\end{rubyblock}

\label{exa:isdg-1}
\end{codeblock}


In order to build a dependency graph, we need to be able to create the solutions. We've decided that each node should consider its two-hop neighborhood. From prior testing, we know that in order to generate sets, the most efficient method is probably going to be to use the {\bfseries Combinator} module, which can take a set of nodes and edges are returns sets of Node ids that represent covers. For this problem, we want to use the module to take a set of nodes and edges and return ids that represent independent sets. Looking at the code in Example~\ref{exa:combinator-1}, we can see the flow of control in the Combinator Module. 

\begin{codeblock}
\caption{Combinator Module}
\begin{LVerbatim}[commandchars=\\\{\},numbers=left,firstnumber=1,stepnumber=1,xleftmargin=7mm, fontsize=\small, boxwidth=3cm]
\PY{k}{module} \PY{n+nn}{Combinator}
  \PY{k}{def} \PY{n+nf}{construct\PYZus{}covers} \PY{n}{n}\PY{p}{,} \PY{n}{e}
    \PY{n}{s} \PY{o}{=} \PY{n}{get\PYZus{}subsets} \PY{n}{n}\PY{o}{.}\PY{n}{collect}\PY{p}{\PYZob{}}\PY{o}{|}\PY{n}{k}\PY{o}{|} \PY{n}{k}\PY{o}{.}\PY{n}{id}\PY{p}{\PYZcb{}}
    \PY{n}{s} \PY{o}{=} \PY{n}{test\PYZus{}covers}\PY{p}{(}\PY{n}{s}\PY{p}{,}\PY{n}{e}\PY{p}{)}
    \PY{k}{return} \PY{n}{covers\PYZus{}to\PYZus{}set} \PY{n}{s}
  \PY{k}{end}

  \PY{k}{def} \PY{n+nf}{get\PYZus{}subsets} \PY{n}{n}
    \PY{n}{subsets} \PY{o}{=} \PY{o}{[}\PY{o}{]}
    \PY{n}{x} \PY{o}{=} \PY{n}{n}\PY{o}{.}\PY{n}{length} \PY{o}{-} \PY{l+m+mi}{1}
    \PY{p}{(}\PY{l+m+mi}{1}\PY{o}{.}\PY{n}{.}\PY{n}{x}\PY{p}{)}\PY{o}{.}\PY{n}{each} \PY{k}{do} \PY{o}{|}\PY{n}{k}\PY{o}{|}
      \PY{n}{subsets} \PY{o}{=} \PY{n}{subsets}\PY{o}{+}\PY{p}{(}\PY{n}{n}\PY{o}{.}\PY{n}{combination}\PY{p}{(}\PY{n}{k}\PY{p}{)}\PY{o}{.}\PY{n}{to\PYZus{}a}\PY{p}{)}
    \PY{k}{end}
    \PY{k}{return} \PY{n}{subsets}
  \PY{k}{end}

  \PY{k}{def} \PY{n+nf}{test\PYZus{}covers}\PY{p}{(}\PY{n}{s}\PY{p}{,} \PY{n}{e}\PY{p}{)}
    \PY{n}{e\PYZus{}array} \PY{o}{=} \PY{o}{[}\PY{o}{]}
    \PY{n}{e}\PY{o}{.}\PY{n}{each}\PY{p}{\PYZob{}}\PY{o}{|}\PY{n}{k}\PY{o}{|} \PY{n}{e\PYZus{}array}\PY{o}{.}\PY{n}{push}\PY{p}{(}\PY{n}{k}\PY{o}{.}\PY{n}{to\PYZus{}a}\PY{p}{)}\PY{p}{\PYZcb{}}
    \PY{n}{c} \PY{o}{=} \PY{o}{[}\PY{o}{]}
    \PY{n}{s}\PY{o}{.}\PY{n}{each}\PY{p}{\PYZob{}}\PY{o}{|}\PY{n}{k}\PY{o}{|} \PY{n}{c}\PY{o}{.}\PY{n}{push}\PY{p}{(}\PY{n}{k}\PY{p}{)} \PY{k}{if} \PY{n}{test\PYZus{}cover?}\PY{p}{(}\PY{n}{k}\PY{p}{,}\PY{n}{e\PYZus{}array}\PY{p}{)}\PY{p}{\PYZcb{}}
    \PY{k}{return} \PY{n}{c}
  \PY{k}{end}

  \PY{k}{def} \PY{n+nf}{test\PYZus{}cover?} \PY{n}{c}\PY{p}{,} \PY{n}{e}
    \PY{n}{e}\PY{o}{.}\PY{n}{each}\PY{p}{\PYZob{}}\PY{o}{|}\PY{n}{k}\PY{o}{|} \PY{k}{return} \PY{k+kp}{false} \PY{k}{if} \PY{p}{(}\PY{n}{c}\PY{o}{&}\PY{n}{k}\PY{p}{)}\PY{o}{.}\PY{n}{empty?}\PY{p}{\PYZcb{}}
    \PY{k}{return} \PY{k+kp}{true}
  \PY{k}{end}    

  \PY{k}{def} \PY{n+nf}{covers\PYZus{}to\PYZus{}set} \PY{n}{covers}
    \PY{n}{l} \PY{o}{=} \PY{n+no}{Set}\PY{o}{[}\PY{o}{]}
    \PY{n}{covers}\PY{o}{.}\PY{n}{each}\PY{p}{\PYZob{}}\PY{o}{|}\PY{n}{k}\PY{o}{|} \PY{n}{l}\PY{o}{.}\PY{n}{add}\PY{p}{(}\PY{n+no}{Set}\PY{o}{.}\PY{n}{new}\PY{p}{(}\PY{n}{k}\PY{p}{)}\PY{p}{)}\PY{p}{\PYZcb{}}
    \PY{k}{return} \PY{n}{l}
  \PY{k}{end}
\PY{k}{end}
\end{LVerbatim}

\label{exa:combinator-1}
\end{codeblock}


The method {\ttfamily construct\_covers} serves as the interface for the other methods. The only method that is specific to returning covers is the method {\ttfamily test\_cover?}, which takes an array of edges and a combination of Node ids and tests whether the combination is a cover. The {\bfseries \&} operator is the set intersection operator, so the line {\ttfamily e.each\{|k| return false if (c\&k).empty?\}} translates as ''return false if one of these edges contains an id that is not in 'c'". For independent set, we want a test that returns false if the id combination (c) contains {\em both} of the ids in any edge. To accomplish this, we could use the array difference operator, and return false if the subtraction of c from any edge results in an empty array. The modified module with the new definition of {\ttfamily test\_cover?} is shown in Example~\ref{exa:is-combinator}.  

\begin{codeblock}[htp]
\caption{Modified IS Combinator}
\begin{rubyblock}
module IS_Combinator
  include Combinator
  include Two_Hop_Complete |; \label{line:two-complete} ;|
  def test_cover? c,e
    e.each{|k| return false if (k-c).empty?}
    return true
  end
end
\end{rubyblock}

\label{exa:is-combinator}
\end{codeblock}


The two referenced modules, {\bf Two-Hop-Star} and {\bf Two-Hop-Complete} (lines \algref{exa:combinator-1}{line:two-star} and \algref{exa:is-combinator}{line:two-complete}), determine specifically which nodes and edges that are passed to the {\ttfamily construct\_covers} method. The node actually calls the method {\ttfamily set\_covers} defined in one of these modules. The module selects a group of nodes and edges, sends them to the {\ttfamily construct\_covers} method, and returns the result.

The new module must be included into the definition of {\bf ISDGRoot} in order to be used, resulting in the definition shown in Example~\ref{exa:isdg-2}. 

\begin{codeblock}[htp]
\caption{Modified Class Definition of ISDG Node}
\begin{rubyblock}
class ISDGRoot < PCDRoot
  include IS_Combinator
  def init_covers
    return IS_Graph.new
  end
  def get_dep_node_type
    return Object.const_get('IS_Graph')
  end
end
\end{rubyblock}

\label{exa:isdg-2}
\end{codeblock}
  

Finally, because this is a two-hop node and the {\bf neighbors} attribute only extends to one hop, it might be useful to add an attribute to contain the two-hop neighborhood. We extend our definition of {\bf ISGRoot} to a new subclass, {\bf ISGNode}, shown in Example~\ref{exa:isdg-node}. We add a readable attribute in line~\ref{line:attr-neighborhood} and define it as an empty array in the overloaded {\ttfamily initialize} method in line~\ref{line:neighborhood}. The keyword {\bf super} in line~\ref{line:isnode-super} indicates that any arguments passed to this method will be passed to the method of the same name in the super-class. 

\begin{codeblock}
\caption{Class Definition of ISDG Node}
\begin{rubyblock}
class ISGNode < ISDGRoot
  attr_accessor :neighborhood |; \label{line:attr-neighborhood} ;|
  def initialize *args
    super  |; \label{line:isnode-super} ;|
    @neighborhood = [] |; \label{line:neighborhood} ;|
  end
  def get_dep_graph_type
    return Object.const_get('IS_Graph')
  end
end
\end{rubyblock}

\label{exa:isdg-node}
\end{codeblock}
 

It makes sense at this time to design the Dependency Graph for Independent Set. The nodes of our dependency graph will be the sets that are built with our modified {\bfseries Combinator} module, and the edges will be determined by the relationships between those nodes. Our first step will be to define the dependency graph itself, with it's constructor, and then we will move on to defining the nodes. Before we design the Dependency Graph for Independent Set, it will be illustrative to look at the root class in Example~\ref{exa:dep-graph-root}.

\begin{codeblock}
\caption{Class Definition of the Dependency Node Root}
\begin{rubyblock}
require 'dep_node'
require 'helpers_netgen'
require 'helpers_dep_node'
require 'set'

class Dep_Graph_Root
  include Neighborly |;\label{code:dep-graph-includes};|
  include Cleanable
  attr_reader :nodes, :edges
  def initialize c, n
    @nodes = []
    @edges = []
    make_nodes c, n
    make_edges
    set_neighbors
  end
  
  def make_nodes c,n
    if !c.empty? then
      c = kill_redundant c |;\label{code:dep-graph-kill};|
      c.each{|k| @nodes.push(get_node_type.new(k,n))} |;\label{code:dep-graph-push-node} ;|
      @nodes.sort!
    end
  end

  def get_node_type
    return Object.const_get('Dep_Node_Root')
  end

  def make_edges 
    e = Set[]
    n = @nodes.dup
    n.length.times do
      a = n.shift
      n.each{|k| if a.ids - k.ids != a.ids then e.add(Set[a.id, k.id]) end}
    end
    @edges = e.to_a
  end  
end
\end{rubyblock}

\label{exa:dep-graph-root}
\end{codeblock}


The {\ttfamily include} statements starting on line~\ref{code:dep-graph-includes} introduce a new module, {\bfseries Cleanable}. Cleanable contains the method {\ttfamily kill\_redundant} called on line~\ref{code:dep-graph-kill}. This method takes the input and deletes any proper subsets. This is useful in our case since every single node in the two-hop neighborhood will be considered an independent set; we would prefer to limit the number of sets to only those sets of maximum size. Figures~\ref{subfig:example-graph} and \ref{subfig:dependency-graph} show a graph and the independent sets that will become nodes in the dependency graph for a specific node after the redundant sets have been pruned. We should also notice the call in line~\ref{code:dep-graph-push-node} for the {\ttfamily get\_node\_type} method. This is a call that we will need to override for our own class definition. 

\begin{figure}[htp]
  \begin{center}
  \subfloat[Weighted Graph $G$]
  {  \begin{tikzpicture}
%      \draw [help lines] (0,-2) grid (4,2);
      \path [help lines] (0,-2) grid (4,2); 
      \node [vx, label=below:30]  (0) at (.5,-1) {0};
      \node [vx, label=above:40] (1) at (0, 0) {1};
      \node [vx, label=below:40] (2) at (1.5,-1.5) {2};
      \node [vx, label=right:70] (3) at (1, 0) {3};
      \node [vx, label=above:10] (4) at (2, 1) {4};
      \node [vx, label=right:30] (5) at (2.5, 0) {5};
      \node [vx, label=above:50] (6) at (1, 1.5) {6};
      \node [vx, label=above:60] (7) at (3, 1) {7};
      \node [vx, label=below:20] (8) at (3,-1) {8};
      \node [vx, label=above:30] (9) at (-.5,-.5) {9};
      \path [draw] 
      (0) -- (1)
      (0) -- (2)
      (0) -- (3)
			(0) -- (9)
      (1) -- (3)
      (1) -- (6)
			(1) -- (9)
      (2) -- (3)
      (2) -- (5)
      (3) -- (4)
      (3) -- (6)
      (4) -- (5)
      (4) -- (6)
      (4) -- (7)
      (5) -- (7)
      (5) -- (8);
    \end{tikzpicture}

	\label{subfig:example-graph}
	}
  \\
	\rule{2cm}{.5pt}\\
  \subfloat[ISD nodes for Vertex 4]{
	    \begin{tikzpicture}
%      \draw [help lines] (0,-3) grid (6,3);
     \path [help lines] (0,0) grid (4,2);
      \node [ex] (S1) at (1,2) {$4,8,2,1$};
      \node [ex] (S2) at (3,2) {$4,8,0$};
      \node [ex] (S3) at (5,2) {$7,8,0,6$};
      \node [ex] (S5) at (1,1) {$7,8,2,1$};
      \node [ex] (S4) at (3,1) {$7,8,2,6$};
      \node [ex] (S6) at (5,1) {$7,8,3$};
      \node [ex] (S7) at (1,0) {$5,3$};
      \node [ex] (S9) at (3,0) {$5,1$};
      \node [ex] (S8) at (5,0) {$6,5,0$};
    \end{tikzpicture}

 	\label{subfig:dependency-graph}
  }
\\	
	\rule{2cm}{.5pt}\\
  \subfloat[Neighborhood of DN $\{7,8,2,1\}$]{
	    \begin{tikzpicture}
%      \draw [help lines] (0,-3) grid (6,3);
     \path [help lines] (0,-2) grid (4,2);
      \node [ex] (S1) at (1,2) {$4,8,2,1$};
      \node [ex] (S2) at (3,2) {$4,8,0$};
      \node [ex] (S3) at (5,2) {$7,8,0,6$};
      \node [ex, label={left: $weight=160$}, label={right: $degree=12$}] (S5) at (3,0) {$7,8,2,1$};
      \node [ex] (S4) at (1,-2) {$7,8,2,6$};
      \node [ex] (S6) at (5,-2) {$7,8,3$};
      \node [ex] (S9) at (3,-2) {$5,1$};

			\path[draw]
			(S5.north) -- node [below left] {3} (S1.south)
			(S5.north) -- node [left] {1} (S2.south)
			(S5.north) -- node [below right] {2} (S3.south)
			(S5.south) -- node [above left] {3} (S4.north)
			(S5.south) -- node [above right] {2} (S6.north)
			(S5.south) -- node [below left] {1} (S9.north);

    \end{tikzpicture}

  \label{fig:is-graph-2}
	}
  \end{center}
  \caption{Independent Sets}
  \label{fig:is-graph-1}
\end{figure}


At this point, we can easily define the class for an Independent Set Dependency Graph by inheriting the root class and overriding the {\ttfamily get\_node\_type} method to return an independent set dependency node. This class will create the nodes, set the edges for the graph, and build a neighbor list for each node. Before we can use this class, we will need to create the dependency node for independent set. To do this, we will start by inheriting the root class for dependency nodes. The code for this class is shown in Example~\ref{exa:dep-node-root}. In order to properly rank our dependency nodes, we need to add a {\bfseries degree} attribute to the class. This will allow us to represent not only the cumulative weight of each set as calculated in line~\algref{exa:dep-node-root}{code:dep-node-weight}, but also to capture the strength of the dependencies between a node and its neighbors. We will need to override two methods to accomplish this, the {\ttfamily initialize} method and the {\ttfamily <=>} method, which defines comparisons between nodes. The complete class is shown in Example~\ref{exa:dep-node-isg}.

\begin{codeblock}
\caption{Class Definition of the Dependency Node Root}
\begin{rubyblock}
class Dep_Node_Root
  @@id = 0
  attr_reader :weight, :ids, :id, :neighbors
  def initialize n, nodes
    @weight = nodes.select{|k| n.include?(k.id)}.collect{|k| k.weight}.inject(0){|a,b| a+b} |;\label{code:dep-node-weight};|
    @neighbors = []
    @ids = n
    @id = @@id
    @@id +=1
  end
  def <=> (b)
    if @weight < b.weight
      return -1
    elsif @weight > b.weight
      return 1
    elsif @id < b.id
      return -1
    else
      return 1
    end
  end
end
\end{rubyblock}

\label{exa:dep-node-root}
\end{codeblock}

\begin{codeblock}
\caption{Class Definition of the IS Dependency Node}
\begin{rubyblock}
require 'dep_node'
class ISG_Graph_Node < Dep_Node_Root
  attr_reader :degree
  def initialize n, nodes
    super
    @degree = 0
  end
  def set_degree d
    @degree = d
  end
  def <=> b
    if @degree < b.degree
      return -1
    elsif @degree > b.degree
      return 1
    elsif @weight < b.weight
      return -1
    elsif @weight > b.weight
      return 1
    elsif @id < b.id
      return -1
    else
      return 1
    end
  end
end
\end{rubyblock}

\label{exa:dep-node-isg}
\end{codeblock}


Setting the degree for each node is best done by the Dependency Graph after edges and neighbors have been set. Each dependency node is a set of nodes from the base graph. A dependency node's neighbors are the dependency nodes which intersect with this set. We define the weight of an edge between two nodes implicitly as the magnitude of this intersection, and the degree of the node as the sum of these weights. Figure~\ref{fig:is-graph-2} shows the neighbors, weights, and degree for the node $\{7,8,2,1\}$. Example~\ref{exa:dep-graph-isg} shows the completed dependency graph class.
\begin{codeblock}
\caption{Class Definition of the IS Dependency Graph}
\begin{rubyblock}
require 'dep_node_isg'
require 'dep_graph'
class ISG_Graph < Dep_Graph_Root
  def initialize
    super
    set_degrees
  end
  def get_node_type
    return Object.const_get('ISG_Graph_Node')
  end
  def set_degrees
    @nodes.each do |k|
      d = 0
      k.neighbors.each do |j|
        d += (k.ids & j.ids).length
      end
      k.set_degree d
    end
  end
end
\end{rubyblock}

\label{exa:dep-graph-isg}
\end{codeblock}
 

To complete the basic structural framework for the new algorithm, we need to create network and simulation generators. Commonly networks are not constructed on their own. In a typical experiment, several algorithms must be run on one network configuration. To this end, most network constructors take a network as their input. This is also the reason that most node types take a node as a constructor argument. In this case, we take advantage of that to write two very short definitions, the Root and Standard classes for IS\_Networks shown in Example~\ref{exa:netgen-isg}.\footnote{It is also important to require the file containing the IS Node definitions in the 'netgen.rb' file.} We also need to add a method to the {\bf Neighborly} module to build the 2-hop neighborhood.

 A similar strategy can be used to create a simulator, simply copying an existing simulator and substituting this network class at the appropriate spot will create a functioning simulator.

At this point, we have defined the basic structure for the framework. We have defined the dependency graph, the relationships between dependency nodes, the nodes themselves, and a network class that will transform a generic network into a network initialized to use this algorithm. Having completed the data structures, we can move on to the algorithm that will use these data structures. 

\begin{codeblock}
\caption{Class Definition of ISDG Network}
\begin{rubyblock}
require 'netgen'
class ISGRootGraph < SimpleGraph
  include Neighborly
  def initialize g
    super()
    @edges = g.edges
    @nodes = []
    put_nodes(g)
    set_neighbors
    set_two_neighborhood
  end
  def put_nodes g
  end
end
class ISGGraph < ISGRootGraph
  def put_nodes g
    g.nodes.each{|k| @nodes.push(ISGNode.new(k))}
  end
end
\end{rubyblock}

\label{exa:netgen-isg}
\end{codeblock}


\paragraph{Actions}
As detailed in Sections~\ref{sec:sim-objects-simulator} and \ref{sec:sim-helpers-simulation}, each simulator has a single graph or network as an attribute, and simulation is accomplished by repeatedly calling the {\ttfamily do\_next} and {\ttfamily send\_status} methods of each node in this network. The {\ttfamily do\_next} method is simply a finite state machine that walks each node through appropriate, synchronized steps. The {send\_status} method usually calls the {\ttfamily receive\_status} method of every neighbor, using the current state of the node to guide what status message is communicated. This information will be used to navigate the dependency graph. 

We must define two automata. One controls the process of the algorithm in the network, the communication of information and the decision making process. The other concerns the internal operation of a node in deciding how to evaluate the nodes on the dependency graph and when to turn on or off. The rules for both are relatively simple. The communication behavior, including the transfer of two hop information, is captured in the module defined in Example~\ref{exa:actions-isg}.  
\begin{codeblock}
\caption{IS Actions}
\begin{rubyblock}
module ISG_Acts
  def do_next
    @now = @next
    case @now
    when :analyze
      if best_choice? |;\label{code:bestchoice-call};|
        @on = true
        @next = :share
      elsif no_choice? |;\label{code:nochoice-call};|
        @on = false
        @next = :share
      else
        @next = :wait
      end
    when :wait
      @next = :update
    when :update
      update_dep_graph |;\label{code:update-call};|
      @next = :analyze
    when :share
      @next = :done
    when :done
      @next = :done
    end      
  end
  def send_status
    case @next
    when :share
      @neighbors.each{|k| k.recieve_status self}
    when :update, :done
      @neighbors.each{|k| k.recieve_status @onlist, @offlist}
    end
  end
  def recieve_status *args
    if args.size == 1 then
      if args[0].on then
        @onlist.add(args[0].id)
      elsif args[0].off?
        @offlist.add(args[0].id)
      end
    elsif args.size == 2 then
      @onlist = @onlist | args[0] |;\label{code:union-1};|
      @offlist = @offlist | args[1] |;\label{code:union-2};|
    end
  end
end
\end{rubyblock}

\label{exa:actions-isg}
\end{codeblock}


Some explanation may be in order about the notation of the Ruby language. The Ruby {\bfseries case} statement can be interpreted as being similar to a switch in C, and when placed between two sets, as in lines~\algref{exa:actions-isg}{code:union-1} and \ref{code:union-2}, the vertical bar denotes set union. The rules enforce the requirement that nodes be able to collect information from a two-hop neighborhood. The {\bfseries :analyze}, {\bfseries :update}, and {\bfseries :done} states correspond to the states \cAd, \cUd, and \cId/\cOd\ from Figure~\ref{fig:isdg-alg}. 

The three undefined methods in lines~\ref{code:bestchoice-call}, \ref{code:nochoice-call}, and \ref{code:update-call} are shown in Example~\ref{exa:deciders-isg}. The {\ttfamily no\_choice?} method checks to see if any neighbors have decided to join the independent set, which would force a node to decide not to join the set itself. The {\ttfamily best\_choice?} and {\ttfamily update\_dep\_graph} methods require modifications to the Dependency Graph class. In particular, the method {\ttfamily update\_dep\_graph} requires that we define the rules for navigating the dependency graph. 
\begin{codeblock}
\caption{Decision Methods for IS Nodes}
\begin{rubyblock}
module ISG_Deciders
  def no_choice?
    return !@neighbors.select{|k| k.on}.empty?
  end
  def best_choice?
    if no_choice?
      return false
    else
      return (@covers.current_cover.include? @id) and (@neighbors.select{|k| !k.off?}.max < self)
    end
  end
  def update_dep_graph
    @covers update_dep_graph @onlist, @offlist
  end
end
\end{rubyblock}

\label{exa:deciders-isg}
\end{codeblock}



