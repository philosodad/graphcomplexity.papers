\begin{note}[Communication Model]
\label{not:com-model}
We assume a message passing model of distributed computing. In each communication round, it is assumed that every node can communicate with its neighbors. Communication is assumed to be synchronous and symmetric: if node $a$ is a neighbor of node $b$, node $b$ is a neighbor of node $a$, and if node $a$ has counted $x$ communication steps, so has node $b$.

A ``communication round'' is actually three steps: an invitation sending step, an information response step, and an exchange step where neighbors share changes in status.\footnote{It should be noted that the prior algorithms in the literature also require three communication events per round.} 
\end{note}
\begin{note}[Local Information]
\label{not:dgmm-local-info}
At the beginning of each communication round, each node has a list of its neighbors, their current state, the edges associated with those neighbors, and the results of any previous computation performed on those edges.
\end{note}
\begin{note}[Mapping to Sequential Algorithm]
\label{not:gmm-dgmm}
DGMM is based on a sequential algorithm (GMM), which takes as an input a graph and produces a 2-optimal vertex cover of that graph. The sequential algorithm selects each edge of the graph in turn, in arbitrary order, and compares the endpoints of that edge. The edge is assigned a weight according to Equation~\ref{eqn:gmm}. If one endpoint is already in the cover, the resulting weight will be zero, otherwise, one endpoint will be added to the cover. When each edge has been assigned a weight, the algorithm terminates and outputs the cover.

For DGMM, the graph is represented as a network of compute nodes, with the nodes representing vertices and connections between the nodes representing edges. Nodes form pairs over these connections and assign weights to each connection following the same rules as the sequential algorithm. A node that joins the cover turns itself "on", while nodes turn off if they have no unweighted edges and have not joined the cover. When the algorithm terminates, every node in the network that is in the cover will be in an ``on'' state, and every node that is not will be in an ``off'' state.
\end{note}
