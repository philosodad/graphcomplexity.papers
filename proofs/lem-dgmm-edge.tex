\begin{lem}
\label{lem:dgmm-edge}
  DGMM weights each edge once in a manner equivalent to GMM.
\end{lem}
\begin{proof}[Proof of Lemma~\ref{lem:dgmm-edge}]
  Lemma~\ref{lem:dgmm-edge} can be restated in terms of the following propositions.
  \begin{lprp}
    \label{prop:dgmm-edge-order}
    Given a matching, a simultaneous weighting of the edges in that matching is equivalent to an arbitrary sequential weighting of the same edges.
  \end{lprp}
  \begin{lprp}
    \label{prop:dgmm-edge-match}
    DGMM produces a matching in each communication round.
  \end{lprp}
  \begin{lprp}
    \label{prop:dgmm-edge-once}
    DGMM weights every edge exactly once.
  \end{lprp}
  If these propositions are true, Lemma~\ref{lem:dgmm-edge} is also true. 
  
  \begin{proof}[Proof of Proposition~\ref{prop:dgmm-edge-order}]
    Equation~\ref{eqn:gmm} for weighting an edge $e(u,v)$ is 
    \begin{equation*}
      weight(e(u,v)) = min
      \begin{dcases}
        weight(u) - \sum_{i \ne v} weight(e(u,i))\\
        weight(v) - \sum_{i \ne u} weight(e(v,i))
      \end{dcases}
    \end{equation*}
    By definition of a matching, no two edges in a matching share a vertex. Therefore, if an edge $e(u,v)$ is in the matching, no edge $e(u,i)$ is in the matching. 
    Take a matching \bMd\ in a Graph $G(V,E)$, composed of edges $\{e_0, e_1, ..., e_n\}$. If we use the sequential algorithm to weight the edges in \bMd\ one after the other, it is obvious that no edge outside of \bMd\ will change. Since only edges outside of \bMd\ are used to assign weights to edges inside \bMd\, it does not matter what order the weights are assigned in, or whether the weight assignment occurs to all edges in \bMd\ simultaneously.
    
    Therefore Proposition~\ref{prop:dgmm-edge-order} is true.
  \end{proof}
  \begin{proof}[Proof of Proposition~\ref{prop:dgmm-edge-match}]
    Assume not, that is, assume that there are two edges $e(u,v) \text{ and } e(u,i)$ that are both updated during the same communication round. For this to happen, some compute node $u$ must form a partnership with two nodes $i$ and $v$. 
    
At the beginning of every communication round, each node makes an equally weighted random decision to either issue an invitation or wait for invitations. We consider these options by cases.

    Case One: Assume that $v$ issues invitations. If $v$ issues invitations, $v$ will choose a single unweighted edge $(v,u)$ and broadcast an invitation with the id of $u$ to all of its neighbors (Line~\algref{alg:dgmm}{alglin:dgmm-issue-invite}). $v$ then transitions to the \cWd\ state. In this state, the node gathers all responses issued by its neighbors, and updates an edge if a response is sent specifically to .

    So if two edges are weighted, $v$ must receive two responses.

    Responses are issued by nodes in the \cRd\ state. Each node in this state chooses a single invitation from its received invitations and responds to it. Since $v$ gets two responses, therefore, $v$ must have invited two separate nodes in this round. But $v$ only issues one invitation, so this is a contradiction.

    Case Two: Assume that $v$ receives invitations. A node which recieves invitations updates the edge corresponding to the invitation it recieved. Since $v$ is weighting two edges, $v$ must respond to multiple invitations in this round. However, $v$ only sends a single response message (Line~\algref{alg:dgmm}{alglin:dgmm-choose-invite}., which is a contradiction as well.
    Therefore, Proposition~\ref{prop:dgmm-edge-match} is true.
  \end{proof}
  \begin{proof}[Proof of Proposition~\ref{prop:dgmm-edge-once}]
    Because a node will only attempt to weight an unweighted edge, we know that no edge will be weighted more than once. If the proposition is false, it must be the case that some edge is not weighted.
    For an edge to be unweighted, both endpoints of the edge would have to halt (enter the \cDd\ state) before the edge is weighted. Nodes halt under two circumstances:
    \begin{enumerate}
    \item The node has joined the cover.
    \item A nodes neighbors have all joined the cover.
    \end{enumerate}
    In the first case, the node will weight all of its unweighted edges to 0. In the second case, the node weights its own edges to 0 if the other endpoint is in the cover.
    Therefore, if the algorithm halts, all edges have been weighted once.
  \end{proof}
  Therefore, Lemma~\ref{lem:dgmm-edge}: our algorithm weights edges equivalently to the GMM algorithm. As we have indicated, prior work by Gonzalez has shown GMM to be 2-optimal, therefore DGMM is two optimal as well.
\end{proof}
